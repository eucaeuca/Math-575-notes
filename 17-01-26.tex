\section{Class Notes 17-01-26}
\begin{theorem}
	$(\Z/p\Z)$ is a field.
\end{theorem}
\begin{proof}
	If $[a]\ne0 \Rightarrow (p,a) = 1 \Rightarrow \exists x,y\ s.t.\ px+ay = 1\Rightarrow [a][y] = [1]$.
\end{proof}
\begin{theorem}
	Let $K$ be a field, let $G$ be a finite subgroup of $K$, then $G$ is cyclic.
\end{theorem}
\begin{lemma}
	Let $f(x)\in K[x]$ be any non-zero polynomial. Then the number of roots of $f$ in $K$ is elss or equal to $\deg{f}$
\end{lemma}
\begin{proof}
If $f(x)$ has no root, we are done. If $f(x)$ has some roots, say $\al$ is a root, then $$f(x) = (x-\al)g(x)+ r(x),\quad r(x) = 00$$So $f(x) = (x-\al)g(x)$. By induction the lemma holds.
\end{proof}
We can then prove the theorem.
\begin{proof}
	Let $K$ be a field. Let $G\subset K^{\times}$ be a finite subgroup of order $n$. $G\subset \{\hbox{roots of} x^n-1\}\Rightarrow G = \{\hbox{roots of} x^n -1\}$. Any element in $G$ has order dividing by $n$ for every divisor $d$ of $n$. Let $\Sigma_d = \{a\in G, o(a) = d\}$, then $$G=\sqcup_{d\mid n} \Sigma_d, \quad n=|G| = \sum_{d\mid n}|\Sigma_d|.$$
	We claim: $|\Sigma_d| = 0$ or $\phi(d)$.\\
	If $\Sigma_d = \emptyset\Rightarrow |\Sigma_d| = 0$. Suppose $\Sigma_d\ne \emptyset\Rightarrow \exists a\in G, \hbox{s.t.} o(a) = d$. Let $H =\langle a\rangle = \{1,a,\ldots, a^{d-1}\}\subset G.$
	i.e.,\begin{align*}\Sigma_d &=\hbox{set of elements with order } d\\
	&= \hbox{all elements of } H
	\end{align*}.
	$\Rightarrow |\Sigma_d| = \phi(d)$. Then $$n=\sum_{d\mid n}|\Sigma_d|\le \sum_{d\mid n}\phi(d) = n$$
	$\Rightarrow |\Sigma_d| = \phi(d), \forall d\mid n$. In particular $|\Sigma_n| = \phi(n)\Rightarrow G$ is cyclic.
\end{proof}
We then want to discuss the structure of $(\Z/p^{\gamma}\Z)^{\times}$
\begin{theorem}
	If $p$ is an odd prime, then $(\Z/p^{\gamma}\Z)^{\times}$ is cyclic.
\end{theorem}
\begin{proof}
	Since $\Z/p^{\gamma}\Z\rightarrow\Z/p\Z$ is surjective, $(\Z/p^{\gamma}\Z)^{\times}\rightarrow(\Z/p\Z)^{\times}$ is surjective. Let us denote $G:=(\Z/p^{\gamma}\Z)^{\times}$, $H:=(\Z/p\Z)^{\times}$, and let $K$ be the kernal of $G\rightarrow H$, i.e.,
	 $$K=\{[x]\in G, x\con{1}{p}\}.$$
	Note we have $|G|=p^{\gamma -1 }(p-1), |H| = p-1$. So we have $|K| = \dfrac{|G|}{|H|} = p^{\gamma -1}$. We will show $K$ is cyclic by explicitly constructing a system. We consider the cyclic group generated by $1+ap$, where $a\con{0}{p}$. We know that 
	$$(1+ap)^{p^{\gamma-1}}\con{1}{p^{\gamma}},$$want however 
	$$(1+ap)^{p^{\gamma-2}}\ncon{1}{p^{\gamma}}.$$
	\begin{lemma}\label{yi}
		Let $p$ be any prime, $a,b\in\Z,\gamma\ge1$. If $a\con{b}{p^{\gamma}}$, then $a^p\con{b^p}{p^{\gamma+1}}$.
	\end{lemma}
	\begin{proof}
		First notice that for $1\le i \le p-1$, $\comb{p}{i}$ is divided by $p$, then
		\begin{align*}
			a=b+p^\gamma t\ &\Rightarrow\ a^p=(b+p^\gamma t)^p\\
			&\Rightarrow\ a^p=b^p+\sum\limits_{i=1}^{p-1}\comb{p}{i}b^i (p^\gamma t)^{p-1} + (p^{\gamma}t)^{p}.\\
			&\Rightarrow\ a^p\con{b^p}{p^{\gamma+1}}
		\end{align*}
	\end{proof}
	We then prove the following lemma,
	\begin{lemma}
		$(1+ap)^{p^{\gamma-2}}\con{1+ap^{\gamma-1}}{p^{\gamma}}$
	\end{lemma}
	\begin{proof}
		We induction on $\gamma$.\\When $\gamma=1$, the statement is trivially true. Assume the statement is true for $\gamma$, check for $\gamma+1$.\\
		We know $$(1+ap)^{p^{\gamma-2}}\con{1+ap^{\gamma -1}}{p^\gamma},$$and we want to show $$(1+ap)^{p^{\gamma-1}}\con{1+ap^{\gamma}}{p^{\gamma+1}}$$By lemma \ref{yi}, \begin{align*}(1+ap)^{p^{\gamma-1}}&\con{(1+ap^{\gamma-1})^p}{p^{\gamma+1}}\\
		&= 1+ p\cdot ap^{\gamma-1}+\sum\limits_{i=2}^{p-1}\comb{p}{i}(ap^{\gamma-1})^i+a^pp^{p(\gamma-1)}\\
		&\con{1+ap^{\gamma}}{p^{\gamma+1}}
		\end{align*}
		So the statement holds for $\gamma+1$.
	\end{proof}
\end{proof}

