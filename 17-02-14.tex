\section{Class Notes 17-02-14}
\underline{Construction of a field of size $p^n$:}
We can find a field $K$ such that $x^{p^n}-x$ splits completely. First notice that $x^{p^n}-x$ has no multiple roots. This is because if we define $f(x):=x^{p^n}-x$, we have $(f(x), f'(x)) = (x^{p^n}-x, p^nx^{p^n-1}-1) = 1$.
\begin{theorem}
Any two field with $p^n$ elements are isomorphic.
\end{theorem}
\begin{proof}
Suppose $\abs{L} = p^n$, I claim: $\exists\alpha\in L$, such that $\mathbb{F}_p(\alpha)=\mathbb{F}_p[\alpha]$. Let $K$ be a field with $p^n$ element, $K=\mathbb{F}_p(\alpha), K^{\times}=\langle\alpha\rangle
$, Let $\phi$ be the homomorphism from $\mathbb{F}_p[x]\rightarrow \mathbb{F}_p[\alpha]$, $\ker{\phi}=(f(x)), f(x)\in \mathbb{F}_p[x]$, $\Rightarrow \frac{\mathbb{F}_p[x]}{(f(x))}\sim \mathbb{F}_p[\alpha]=\mathbb{F}_p(\alpha)=K,\Rightarrow f(x)$ is an irreducible ($K$ is a field).$$[K:\mathbb{F}_p]=\deg{f}.$$ $\alpha$ satisfies $x^{p^n}-x=0$. Because of the isomorphism, $\alpha$ satisfies $f(x)$. $\Rightarrow$ $\alpha$ satisfies the gcd. But $\gcd(x^{p^n}-x, f(x)) =f(x)\Rightarrow f(x)\mid (x^{p^n}-x)$
\end{proof} 
So far, we proved, $\exists\alpha$, such that $K=\mathbb{F}_p(\alpha)$.
\begin{enumerate}
	\item any such $\alpha$ is a root of an irreducible polynomial in $\mathbb{F}_p[x]$ that divides $x^{p^n}-x$.
	\item $K\sim \frac{\mathbb{F}_p[x]}{f(x)}$ for some irreducible $f(x)$, $f(x)\mid (x^{p^n}-x)$.
	\item Let $g(x)$ be any irreducible factor of $x^{p^n}-x$ of degree $n$, $K' = \frac{\mathbb{F}_p}{(g(x))}$, we know $[K': \mathbb{F}_p]=n$, $\Rightarrow x^{p^n}-x$ splits completely in $K'$, $\Rightarrow$ $f$ has a root $\beta\in K'$.
\end{enumerate}
\begin{exercise}
How many monic irreducible polynomial of $\deg 9$ over $\mathbb{F}_7$? 
\end{exercise}
\begin{proof}
Consider the finite extension of the field $\mathbb{F}_7$. Over the field $\mathbb{F}_{7^9}$, $x^{7^9}-x$ splits completely. So$$ x^{7^9}-x=(x^{7^3}-x)* \prod_{\deg{g}=9, \mbox{irr over }\mathbb{F}_7} g,$$ So ans$=\frac{7^9-7^3}{9}$.
\end{proof}
We now want to prove the quadratic reciprocity using finite field. i.e., we want to prove
$$\leg{p}{q}\leg{q}{p}=(-1)^{\frac{p-1}{2}\frac{q-1}{2}}$$
where $p,q$ are odd primes.
\begin{proof}
We can find $n$, such that $q^n\con{1}{p}$. Consider the multiplicative subgroup $\F_{q^n}^\times$. Say $\gamma$ is a generator of this subgroup, let $\lambda = \gamma^{\frac{q^n-1}{p}}$, then order of $\lambda$ is $p$. Let $\tau_a = \sum_{i=1}^{p-1}\leg{i}{p}\lambda^{ai}$ and denote $\tau = \tau_1$, we claim
\begin{enumerate}
\item
$\tau_a=\leg{a}{p}\tau$,
\item
$\tau^2 = (-1)^{\frac{p-1}{2}}p$.
\end{enumerate}
For the first claim, assume $\gcd(p,a)=1$, let $ab\con{1}{p}$, we notice that 
\begin{align*}
\tau_a &= \sum_{i=1}^{p-1}\leg{i}{p}\lambda^{ai}\\
&= \sum_{i=1}^{p-1}\leg{bi}{p}\lambda^{abi}\\
&= \sum_{i=1}^{p-1}\leg{b}{p}\leg{i}{p}\lambda^{i}\\
&= \leg{b}{p} \tau = \leg{a}{p} \tau
\end{align*}
For the second claim, we have 
\begin{align*}
\tau^2 &= \left(\sum_{i=1}^{p-1}\leg{i}{p}\lambda^{i} \right)\left(\sum_{i=1}^{p-1}\leg{i}{p}\lambda^{i} \right)\\
&= \sum_{i=1}^{p-1}\sum_{j=1}^{p-1}\leg{i}{p}\leg{j}{p}\lambda^{i+j}\\
&= \sum_{i,j}\leg{ij}{p}\lambda^{i+j}\\
&= \sum_{i,j}\leg{i^2j}{p}\lambda^{i+ij}\\
&= \sum_{i,j}\leg{j}{p}\lambda^{i(j+1)}\\
&= \sum_{j=1}^{p-1} \leg{j}{p}\sum_{i=1}^{p-1}\lambda^{i(j+1)}
\end{align*}
But $\sum_{i=1}^{p-1}\lambda^{i(j+1)}=\left\{\begin{matrix}
p-1, & j=-1\\ -1, & j\ne-1\end{matrix}\right.$, So \begin{align*}
\tau^2 &=\sum_{j\ne-1}\leg{j}{p}(-1)+\leg{-1}{p}(p-1)\\
&=\leg{-1}{p}p-\sum_{j}\leg{j}{p}\\
&=\leg{-1}{p}p.
\end{align*}
Denote $p^*:=\leg{-1}{p}p$,
\begin{align*}
\tau=\sum_j\leg{j}{p}\lambda^j & \Rightarrow \tau^q=(\sum_j\leg{j}{p}\lambda^j)^q\\
& \Rightarrow \tau^q = \sum_{j}\leg{j}{p}\lambda^{jq} = \tau_q = \leg{q}{p}\tau
\end{align*}
Thus $\leg{q}{p}=1$ iff $\tau^q=\tau$.
\begin{align*}
\leg{q}{p}=1 & \Leftrightarrow \tau^q=\tau \Leftrightarrow \tau\in\F_q\\
& \Leftrightarrow p^*\mbox{ is a square in } \F_q \Leftrightarrow \leg{p^*}{q}=1
\end{align*}
Thus $$\leg{q}{p}\leg{p*}{q}=1$$
or $$\leg{p}{q}\leg{q}{p}=(-1)^{\frac{p-1}{2}\frac{q-1}{2}}$$
\end{proof}