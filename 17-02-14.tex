\section{Class Notes 17-02-14}
\underline{Construction of a field of size $p^n$:}
We can find a field $K$ such that $x^{p^n}-x$ splits completely. First notice that $x^{p^n}-x$ has no multiple roots. This is because if we define $f(x):=x^{p^n}-x$, we have $(f(x), f'(x)) = (x^{p^n}-x, p^nx^{p^n-1}-1) = 1$.
\begin{theorem}
Any two field with $p^n$ elements are isomorphic.
\end{theorem}
\begin{proof}
Suppose $\abs{L} = p^n$, I claim: $\exists\alpha\in L$, such that $\mathbb{F}_p(\alpha)=\mathbb{F}_p[\alpha]$. Let $K$ be a field with $p^n$ element, $K=\mathbb{F}_p(\alpha), K^{\times}=\langle\alpha\rangle
$, Let $\phi$ be the homomorphism from $\mathbb{F}_p[x]\rightarrow \mathbb{F}_p[\alpha]$, $\ker{\phi}=(f(x)), f(x)\in \mathbb{F}_p[x]$, $\Rightarrow \frac{\mathbb{F}_p[x]}{(f(x))}\sim \mathbb{F}_p[\alpha]=\mathbb{F}_p(\alpha)=K,\Rightarrow f(x)$ is an irreducible ($K$ is a field).$$[K:\mathbb{F}_p]=\deg{f}.$$ $\alpha$ satisfies $x^{p^n}-x=0$. Because of the isomorphism, $\alpha$ satisfies $f(x)$. $\Rightarrow$ $\alpha$ satisfies the gcd. But $\gcd(x^{p^n}-x, f(x)) =f(x)\Rightarrow f(x)\mid (x^{p^n}-x)$
\end{proof} 
So far, we proved, $\exists\alpha$, such that $K=\mathbb{F}_p(\alpha)$.
\begin{enumerate}
	\item any such $\alpha$ is a root of an irreducible polynomial in $\mathbb{F}_p[x]$ that divides $x^{p^n}-x$.
	\item $K\sim \frac{\mathbb{F}_p[x]}{f(x)}$ for some irreducible $f(x)$, $f(x)\mid (x^{p^n}-x)$.
	\item Let $g(x)$ be any irreducible factor of $x^{p^n}-x$ of degree $n$, $K' = \frac{\mathbb{F}_p}{(g(x))}$, we know $[K': \mathbb{F}_p]=n$, $\Rightarrow x^{p^n}-x$ splits completely in $K'$, $\Rightarrow$ $f$ has a root $\beta\in K'$.
\end{enumerate}
\begin{exercise}
How many monic irreducible polynomial of $\deg 9$ over $\mathbb{F}_7$? 
\end{exercise}
\begin{proof}
Consider the finite extension of the field $\mathbb{F}_7$. Over the field $\mathbb{F}_{7^9}$, $x^{7^9}-x$ splits completely. So$$ x^{7^9}-x=(x^{7^3}-x)* \prod_{\deg{g}=9, \mbox{irr over }\mathbb{F}_7} g,$$ So ans$=\frac{7^9-7^3}{9}$.
\end{proof}
We now want to prove the quadratic reciprocity using finite field.
\begin{proof}
\notcomplete
\end{proof}