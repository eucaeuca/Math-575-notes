\section{Class Notes 17-01-12}\label{PIDUFD}
\begin{definition}
A non-zero element in \R{} is called a unit if $\exists\, v\in \R$ such that $uv=1_\R$.
\end{definition}
\begin{definition}
Two element $a,b \in\R$ are said to be associative if $\exists\, u\in\R$, $u$ is a unit, such that $a=bu$, denoted by $a\sim b$.
\end{definition}
\begin{definition}
A non-zero element $\pi$ in \R{} is said to be irreducible if $\pi$ is not a unit and if $a\mid \pi\Rightarrow$ $a$ is a unit or $a$ is associative of $\pi$.
\end{definition}
\begin{definition}
A non-zero element in $\R$ is said to be prime if $\pi$ is not a unit and $\pi\mid  ab\Rightarrow \pi\mid a$ or $\pi\mid b$, $\forall a,b\in\R.$
\end{definition}
\begin{proposition}
If $\pi$ is a prime, then $\pi$ is irreducible.
\end{proposition}
\begin{proof}
Let $\pi$ be a prime, suppose $a\mid \pi$, then $\pi=ab$ for some $b\in\R$. Thus $\pi\mid ab$ and by definition, $\pi\mid a$ or $\pi\mid b$.
\begin{itemize}
\item If $\pi\mid a$, then $a\sim \pi$.
\item If $\pi\mid b$, then $a\sim 1$.
\end{itemize}
\end{proof}
\begin{remark}
A irreducible is not necessary to be a prime.\\
Let $R = \Z[\sqrt{5}] = \{a+b\sqrt{-5}\ \mid \ a,b\in\Z\}\subset \C$. We have $$6=2\cdot3 = (1+\sqrt{-5})\cdot(1-\sqrt{-5}).$$We write $\pi = (1+\sqrt{-5})$ and claim that $2,3,\pi,\overline{\pi}$ are irreducibles but none of them are associative of each other.\\
We define the norm function $N:R\rightarrow \Z$, where $N(\al)=\al\overline{\al}$, i.e., if $\al = a+bi$, then $N(\al)=a^2+5b^2$. We notice that 
\begin{itemize}
\item If $\al>0$, then $N(\al)>0$.
\item $N(\al\be)=N(\al)N(\be)$.
\end{itemize}
\underline{Check: $2$ is irreducible:}\\
\underline{Find unit:} \\$N(uv) = N(1) = 1 = N(u)N(v) \Rightarrow N(u)=N(v) = 1$. But $a^2+5b^2 = 1\Rightarrow a=\pm1, b=0$.\\Suppose $2=\al\be$, then $4=N(2) = N(\al\be) = N(\al)N(\be)$.
\begin{enumerate}
\item If $N(\al) = 1, N(\be) = 4$\\
Then $\al$ is a unit $\Rightarrow$ $2$ is irreducible.
\item If $N(\al) = 2, N(\be) = 2$\\
Then $a^2+5b^2 = 2$ has no solution.
\end{enumerate}
\end{remark}
\begin{definition}
An UFD (Unique Factorization Domain) is an integral domain $R$ in which every non-zero element(up to unit) factors uniquely into a product of irreducibles.
\end{definition}
\begin{proposition}
Let $R$ be a domain in which factorization (of irreducibles) exists. Then \emph{$R$ is a UFD $\Leftrightarrow$ every irreducible in $R$ is prime.}
\end{proposition}
\begin{proof}\mbox{}\\
$``\Leftarrow":$ Let $a$ be an element of $R$ and $a\ne0$. If $a=\pi_1\pi_2\cdots\pi_n=\sigma_1\sigma_2\cdots\sigma_m$ are two factorizations. Since $\pi_1$ is prime, $\pi_1\mid \sigma_i$ for some $i$. By rearranging, we may assume $\pi_1\mid \sigma_1$, Thus $\pi_1\sim \sigma_1$. Repeating this process, we can conclude that the two factorizations are the same.
\notcomplete
\end{proof}
\begin{remark}
There are clearly rings such that no factorization exists. For example, consider the ring $\Z[2^{1/2},2^{1/4},2^{1/8},\ldots]\subset \R$. It's the smallest subring of $\R$ that contains $2^{1/2},2^{1/4},\ldots$.
\end{remark}
\begin{definition}
A ring $R$ is said to be noetherian if it satisfies any of the following equivalent conditions:
\begin{enumerate}
\item Any ascending chain of ideals in $R$ terminates.\\
Namely, $I_1\subset I_2\subset I_3\subset \cdots\Rightarrow I_n = I_{n+1}=\cdots$ for some $n$.
\item Any ideal $I$ in $R$ is finite generated.\\
Namely, $I=(a_1,\ldots,a_n)$ for some $n$.
\end{enumerate}
\end{definition}
\begin{proof}
\mbox{}\\\mbox{``1.\ $\Rightarrow$2.\ ": } Let $I$ be an ideal, if $I\ne0$, pick $a_1\in I, a_1\neq 0$, clearly $(a_1)\subset I$. 
If $(a_1)=I$, we are done, If not,
$\exists a_2\in I\backslash(a_1)\Rightarrow (a_1,a_2)\subset I $, this chain terminates.\\
\mbox{``1.\ $\Leftarrow$2.\ ": }Suppose $I_1\subset I_2\subset \ldots$ be an ascending ideal. Let $I = \cup I_n$, we claim that $I$ is an ideal.\\
Let $a,b\in I$, then there exists $n$ such that $a,b\in I_n$. Therefore $a+b\in I_n$, and $a+b\in I$. Let $a\in I$, then $a\in I_n$ for some $n$. Therefore $ra\in I_n\implies ra\in I$. Thus $I$ is an ideal. But $I=(a_1,\ldots,a_m)$, so there exists $n$, such that $a_1,\ldots,a_m\in I_n$. Thus $I=I_n$ and $I_n=I_{n+1}=\cdots$.
\end{proof}
\begin{exercise}
Suppose $R$ is a Noetherian domain, show $R$ admits factorizations.
\end{exercise}
\begin{proof}
If $b$ is not irreducible, then $b=ac$ or $(b)\subset(a)$\notcomplete
\end{proof}
\begin{definition}
A PID (Principle Ideal Domain) is a domain in which every ideal is generated by a single element.
\end{definition}
\begin{theorem}
Every PID is a UFD.
\end{theorem}
\begin{proof}
Let $R$ be a PID, then it's noetherian. So factorizations exist. So it suffices to show that every irreducible is a prime. Let $\pi$ be a irreducible in $R$. Suppose $\pi\mid ab$ and $a$ is not divided by $\pi$. We look at $I=(a,\pi)$, there exists $c\in R$, such that $I=(c)$. Thus we have $c\mid \pi, c\mid a$. So $c\sim 1$ or $c\sim \pi$. Since $c$ is not associative of $\pi$, $c$ is associative of $1$. But then $$1=ax+\pi y$$ for some $x,y\in R$. So $b=abx+\pi by$ or $\pi\mid b$.
\end{proof}

