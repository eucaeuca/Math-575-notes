\section{Unique Factorization in $\mathbb{Z}$}
It will be more convenient to work with $\mathbb{Z}$ rather than restricting ourselves to the positive integers. The notion of divisibility carries over with no difficulty to $\mathbb{Z}$. If $p$ is a positive prime, $-p$ will also be a prime. We shall not consider $1$ or $-1$ as primes even though they fit the definition. This is simply a useful convention. They are called the units of $\mathbb{Z}$.\\There are a number of simple properties of division that we shall simply list.
\begin{enumerate}
\item $a\mid a,a\neq0$.
\item If $a\mid b$ and $b\mid a$, then $a=\pm b$.
\item If $a\mid b$ and $b\mid c$, then $a\mid c$.
\item If $a\mid b$ and $a\mid c$, then $a\mid (b+c)$.
\end{enumerate}
\begin{lemma}
Every nonzero integer can be written as a product of primes.
\end{lemma}
\begin{theorem}
For every nonzero integer $n$ there is a prime factorization$$n=(-1)^{\varepsilon(n)}\prod_pp^{a(p)},$$with the exponents uniquely determined by $n$. In fact, we have $a(p)=\mbox{ord}_pn$.
\end{theorem}
The proof if this theorem if is not as easy as it may seem. We shall postpone the proof until we have established a few preliminary results.
\begin{lemma}
If $a,b\in\mathbb{Z}$ and $b\ge0$, there exist $q,r\in\mathbb{Z}$ such that $a=qb+r$ with $0\leq r<b$.
\end{lemma}
\begin{definition}
If $a_1,a_2,\ldots,a_n\in\mathbb{Z}$, we define $(a_1,a_2,\ldots,a_n)$ to be the set of all integers of the form $a_1x_1+a_2x_2+\cdots+a_nx_n$ with $x_1,x_2,\ldots,x_n\in\mathbb{Z}$.
\end{definition}
\begin{remark}
Let $A = (a_1,a_2,\ldots,a_n)$. Notice that the sum and difference of two elements in $A$ are again in $A$. Also, if $a\in A$ and $r\in\mathbb{Z}$, then $ra\in A$, i.e., $A$ is an ideal in the ring $\mathbb{Z}$
\end{remark}
\begin{lemma}
If $a,b\in \mathbb{Z}$, then there is a $d\in \mathbb{Z}$ such that $(a, b)=(d)$
\end{lemma}
\begin{definition}
Let $a,b\in\mathbb{Z}$. An integer $d$ is called a greatest common divisor of $a$ and $b$ if $d$ is a divisor of both $a$ and $b$ and if every other common divisor of $a$ and $b$ divides $d$.
\end{definition}
\begin{remark}
The gcd of two numbers, if it exists, is determined up to sign.
\end{remark}
\begin{lemma}
Let $a,b\in\mathbb{Z}$. If $(a,b) = (d)$ then $d$ is a greatest common divisor of $a$ and $b$.
\end{lemma}
\begin{definition}
We say that two integers $a$ and $b$ are relatively prime if the only common divisors are $\pm1$, the units.
\end{definition}
It's fairly standard to use the notation $(a, b)$ for the greatest common divisor of $a$ and $b$. With this convention we can say that $a$ and $b$ are relatively prime if $(a,b)=1$.
\begin{proposition}
Suppose that $a\mid bc$ and that $(a,b)=1$. Then $a\mid c$.
\end{proposition}
\begin{corollary}
If $p$ is a prime and $p\mid bc$, then either $p\mid b$ or $p\mid c$.
\end{corollary}
\begin{corollary}
Suppose that $p$ is a prime and that $a,b\in\mathbb{Z}$. Then $\mbox{ord}_pab=\mbox{ord}_pa+\mbox{ord}_pb.$
\end{corollary}

