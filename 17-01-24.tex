\section{Class Notes 17-01-24}
Suppose $I,J\subset R$ are two ideals, how to make new ideals with $I,J$? Evidently, $I\cap J$ and $I+J$ are ideals. Also,
$$I\cdot J:=\{\sum a_i b_i, a_i\in I, b_i\in J\}\subset I\cap J$$ is an ideal. 
\begin{example}
Let $I=m\Z$,$J=n\Z$. then we have \quad\begin{tabular}{|c|c|c|}\hline
$I+J$ & $I\cap J$ & $I\cdot J$\\\hline
$((m,n))$ & $([m,n])$ & $mn\Z$\\\hline
\end{tabular}.
\end{example}
\begin{definition}
We say two ideals $I,J$ are coprime if $I+J=(1)$.
\end{definition}
\begin{remark}
If $I, J$ are coprime, then $I\cap J = I\cdot J$.
\end{remark}
\begin{proof}
For some $x\in I\cap J$, since $I,J$ are coprime, there exists some $a\in I, b\in J$ such that $a+b = 1$. But then $a\cdot x + x\cdot b = x\in I\cdot J$. So $I\cap J \subset I\cdot J$. The other direction is obvious.
\end{proof}
\begin{theorem}[Generalized Chinese Remainder Theorem]
Let $I_1, I_2, \ldots, I_n$ be pairwise coprime ideals in $R$, then the map
$$\phi: R\rightarrow R/I_1\times \ldots \times R/I_n$$
\begin{itemize}
\item[1)] is surjective
\item[2)] has $\ker\phi = I_1I_2\ldots I_n = I_1\cap I_2\cap \ldots \cap I_n$
\end{itemize}
\end{theorem}
\begin{lemma}
We first look at $n=2$ case. If $I,J$ are coprime ideals in $R$, then the map
$$\phi: R\rightarrow R/I \times R/J$$
\begin{itemize}
\item[1)] is surjective.
\item[2)] has $\ker\phi = I\cap J = IJ$.
\end{itemize}
\end{lemma}
\begin{proof}
It's enough to solve the system of congrence
\begin{align*}
x&\con{1}{I}\\
x&\con{0}{J}
\end{align*} and \begin{align*}
y&\con{0}{I}\\
y&\con{1}{J}
\end{align*}
Since $I,J$ are coprime, there exists $c\in I, d\in J$ such that $c+d = 1$. $c, d$ is the solution to our two systems.
\end{proof}
\begin{lemma}
$I_1$ is coprime to $I_2I_3\cdots I_n$.
\end{lemma}
\begin{proof}
There exist \begin{align*}
a_2+b_2 &= 1\\
a_3+b_3 &= 1\\
\ldots\\
a_n+b_n &=1,
\end{align*}
$a_i\in I_1$, $b_j\in I_j$.\\
Then \begin{align*}
b_2b_3\ldots b_n &= (1-a_2)\ldots (1-a_n)\\
&= 1 + a, 
\end{align*}
where $a\in I_1$.\\
By $n=2$ case 
\begin{diagram}
R & \rTo & R/I_1\times R/(I_2\times I_3\times\cdots\times I_n)\\
\dTo & \ruTo\\
R/I_1\times R/I_2\times \cdots\times R/I_n\\
\end{diagram}
\end{proof}
Let us denote $U(R)$ by $R^{\times}$. Note that $\phi(n)=\abs{(\Z/n\Z)^{\times}}$. We now want to look at the structure of $(\Z/n\Z)^{\times}$. We first develop some background in abstract algebra.
\begin{theorem}[Lagrange Theorem]
	Let $G$ be a finite group, $H\subset G$ is a subgroup, then the order of $H$ divides the order of $G$, i.e.,$$|H|\mid|G|$$
\end{theorem}
\begin{proof}
	Take two cosets in $H$, $Ha$ and $Hb$. They are equal or disjoint. So $$|G| = |H|\cdot\hbox{\# of cosets}$$
\end{proof}
\begin{definition}
	If $a\in G$, then $o(a) = \hbox{smallest positive integer}$ $d$ such that $$a^d = 1$$is called the order of the element $a$.
\end{definition}
\begin{corollary}
	$\forall a\in G$, we have $o(a)\mid |G|$.
\end{corollary}
\begin{proof}
	$\langle a\rangle:=\{1,a,\ldots, a^{d-1}\}$ is the subgroup generated by $a$. Then $\langle a\rangle\subset G\Rightarrow d\mid |G|$.
\end{proof}
\begin{corollary}
	$a^{|G|} = 1$.
\end{corollary}
\begin{corollary}
	If $n\ge1$, $(a,n) = 1$, then $a^{\phi(n)}\con{1}{n}$.
\end{corollary}
\begin{proof}
	$(a,n) = 1 \Rightarrow a\rightarrow[a]$ is a unit in $\Z/n\Z$, i.e., $[a]\in(\Z/n\Z)^{\times},|(\Z/n\Z)^{\times}| = \phi(n).$
	$\Rightarrow [a]^{\phi(n)}=1$ in $(\Z/n\Z)^{\times}$,
	i.e., $a^{\phi(n)}\con{1}{n}$.
\end{proof}
\begin{exercise}
	Find the last $3$ digits of $3^{1203}$.
\end{exercise}
\begin{proof}
	$\phi(1000) = \phi(2^3 5^3) = (8-4)(125-25) = 400$. So $3^{400}\con{1}{1000}$. The last three digits are then $027$.
\end{proof}
We now look at the structure of $(\Z/p\Z)^{\times}$, where $p$ is a prime.
\begin{theorem}
	$(\Z/p\Z)^{\times}$ is cyclic.
\end{theorem}
We do some checking, let $p=5,7,11,13$. For $p=11$, we find that $2,3,7,9$ are $\Z/11\Z$'s generator.
\begin{lemma}
	Let $a\in G$ be an element of order $d$, then the order of $a^m$ is $\frac{d}{(d,m)}$.
\end{lemma}
\begin{proof}
	Let $(d,m) = b$, we then have $d = bd', m = bm'$, where $(d',m') = 1$. We claim that $o(a^m) = d'$. For $(a^m)^{d'} \cong a^{bm'd'}\cong a^{dm'}\cong (a^d)^{m'}\cong 1$. Suppose $(a^m)^{l} = 1\Rightarrow a^{ml} = 1\Rightarrow d\mid ml \Rightarrow bd'\mid bm'l \Rightarrow d'\mid m' l\Rightarrow d'\mid l$.
\end{proof}
\begin{corollary}
	If $G$ is cyclic of order $d$, then the number of generators of $G$ is $\phi(d)$.
\end{corollary}

