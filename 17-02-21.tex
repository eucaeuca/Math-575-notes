\section{Class Notes 17-02-19}
\begin{definition}
An algebraic number is an element $\al\in\C$ that is algebraic over \Q. The set of all algebraic numbers will be denoted by $\overline{\Q}$. An algebraic integer is an element $\al\in\C$ satisfies a monic polynomial with coefficients in \Z.
\end{definition}
\begin{exercise}
Which rational numbers are algebraic integer?
\end{exercise}
\begin{proof}
Elements of \Z are clearly algebraic numbers. If $a\not\in\Z,a\in\Q$ and $a$ is an algebraic integer. Suppose $a=\frac{p}{q}$, where $p,q$ are coprime. Then there exists some coefficients $\{b_i\in\Z\}$ such that
$$
a^n + b_{n-1}a^{n-1} + \cdots + b_1 a+ b_0 =0
$$
or
$$
\left(\frac{p}{q}\right)^n + b_{n-1}\left(\frac{p}{q}\right)^{n-1} + \cdots + b_1 \frac{p}{q}+ b_0 =0,
$$But then we have 
$$
p^n + b_{n-1}p^{n-1}q + \cdots + b_1 p q^{n-1}+ b_0 q^{n}=0,
$$
and thus $q\mid p$, contradiction.
\end{proof}
\begin{theorem}
All algebraic integers form a subring.
\end{theorem}
\begin{proof}
Use the same technique we used to prove all algebraic numbers form a field.
\end{proof}
Let $\Omega$ denotes the subring of algebraic integers. Say $\al, \be, \gamma\in \Omega$, we will say $\al\con{\be}{\gamma}$ if $\frac{\al-\be}{\gamma}\in\Omega$. We see that this is just a natural generalization of congrence over \Z.\\
We now provide a proof of $\leg{2}{p}=\left\{
\begin{matrix}
1 & \pm1\con{p}{8} \\ -1 & \pm3\con{p}{8}\end{matrix}\right.$.
\begin{proof}
Let $\xi = e^{\frac{2\pi}{8}}$ be a primitive $8_{th}$ root of $1$. Then we have
$$
0 = \xi^8-1  = (\xi^4-1)(\xi^4+1)
$$
and $\xi^4+1=0$. So $(\xi+\xi^{-1})^2=2$. Let $\tau:=\xi+\xi^{-1}$. $\tau$ is an algebraic integer. (Because of the fact that algebraic integer forms a ring and $\xi^{-1} = \overline{\xi}$ is also a primitive root and thus an algebraic integer). We have
\begin{align*}
\tau^{p-1} &=(\tau^2)^{\frac{p-1}{2}}\\&=2^{\frac{p-1}{2}}\\&=\leg{2}{p}
\end{align*}
So $$\tau^p\con{\leg{2}{p}\tau}{p},$$but $$\tau^{p} = (\xi+\xi^{-1})^{p}= \xi^p+\xi^{-p},$$ 
So if $p\con{\pm1}{8}$, then $\tau^p={\tau}$, so $\tau =\leg{2}{p}\tau\Rightarrow \tau^2 = \leg{2}{p}\tau^2\Rightarrow \leg{2}{p}=1$.\\If $p\con{\pm3}{8}$, then $\tau^p=-\tau\Rightarrow -\tau =\leg{2}{p}\tau\Rightarrow -\tau^2 = \leg{2}{p}\tau^2\Rightarrow \leg{2}{p}=-1$.
\end{proof}
We then give a prove of $\leg{p}{q}\leg{q}{p}=(-1)^{\frac{p-1}{2}\frac{q-1}{2}}$.
\begin{proof}
Let $\xi$ be the primitive $p_{th}$ root of $1$. Let $\tau_a:=\sum_{i=1}^{p-1}\leg{i}{p}\xi^{ai}$. We have claimed, \begin{enumerate} 
\item
$\tau_a=\leg{a}{p}\tau$,
\item
$\tau^2 = (-1)^{\frac{p-1}{2}}p=:p^*$.
\end{enumerate}
Denote $\tau_1$ by $\tau$, then
\begin{align*}
\tau^{q-1} &= (\tau^2)^{\frac{p-1}{2}}\\ &=(p^*)^{\frac{q-1}{2}}\con{\leg{p^*}{q}}{q}
\end{align*}
So $$\tau^{q}\con{\leg{p^*}{q}\tau}{q}.$$
But $\tau=\sum\leg{i}{p}\xi^i\Rightarrow \tau^q = \sum\leg{i}{p}\xi^{iq}=\leg{q}{p}\tau.$ We then have
$$\leg{p^*}{q}\tau \con{\leg{q}{p}\tau}{q}$$
or $$\leg{p^*}{q} \con{\leg{q}{p}}{q}$$
\begin{align*}
\leg{p^*}{q}&=\leg{(-1)^{\frac{p-1}{2}}p}{q}\\
&={\leg{-1}{q}}^{\frac{p-1}{2}}\leg{p}{q}\\
&=\leg{p}{q}(-1)^{\frac{p-1}{2}\frac{q-1}{2}}
\end{align*}
\end{proof}