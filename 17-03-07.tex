\section{Class Notes 17-03-07}
\begin{definition}
A multiplicative character on $\F_p$ is a map $\chi$ from $\F_p^\times$ to the nonzero complex numbers that satisfies 
\begin{align*}
\chi: \F_p^\times &\rightarrow \C^\times\\
\chi(ab) = &\chi(a)\chi(b)\quad\mbox{for all }a,b\in\F_p^\times
\end{align*}
\end{definition}
\begin{example}
Let $p=5$, Find all multiplicative characters modulo $p$.
\end{example}
\begin{proof}
\begin{table}[htbp]
\centering
\begin{tabular}{c|cccc}
  & 1 & 2 & $4(2^2)$ & $3(2^3)$ \\\hline
  $\e$ & 1 & $1$ & $1$ & $1$ \\
  $\chi_1$ & $1$ & $-1$ & $1$ & $-1$\\
  $\chi_2$ & $1$ & $i$ & $-1$ & $-i$\\
  $\chi_3$ & $1$ & $-i$ & $-1$ & $i$\\
\end{tabular}
\caption{The characters of the group $\F_5$\label{char5}}
\end{table}
\end{proof}
\begin{proposition}
Let $\chi$ be a multiplicative character and $a\in\F_p^\times.$ Then
\begin{enumerate}
	\item
	$\chi(1)=1.$
	\item
	$\chi(a)$ is a $(p-1)_{st}$ root of unity.
	\item
	$\chi(a^{-1})=\chi(a)^{-1}=\overline{\chi(a)}.$
\end{enumerate}
\end{proposition}
\begin{proof}\mbox{}
\begin{enumerate}
	\item
	We have $\chi(1)=\chi(1^2)=(\chi(1))^2$, so $\chi(1)=0$ or $\chi(1)=1$. But $\chi(1)$ can't be zero, thus $\chi(1)=1$.
	\item
	$(\chi(a))^{p-1}=\chi(a^{p-1})=\chi(1)=1$.
	\item
	$1=\chi(aa^{-1})=\chi(a)\chi(a^{-1})$, 
	so $\chi(a^{-1})=\chi(a)^{-1}$. Also, since $1=\abs{\chi(a)}=\chi(a)\overline{\chi(a)}$, $\overline{\chi(a)}=(\chi(a))^{-1}$.
\end{enumerate}
\end{proof}
\begin{proposition}Let $\chi$ be a multiplicative character.
\begin{enumerate}
	\item If $\chi\ne\varepsilon$, then $\sum_t\chi(t)=0$, where the sum is over all $t\in\F_p$.
	\item If $\chi=\varepsilon$, then $\sum_t\varepsilon(t) = p$, the sum is over all $t\in\F_p$.
\end{enumerate} 
\end{proposition}
\begin{proof}\mbox{}
\begin{enumerate}
	\item
	if $\chi\ne\varepsilon$, there exist some $a\in\F_p^\times$ such that $\chi(a)\ne 1$. Then $\sum_t\chi(t) =\sum_t\chi(at) = \sum_t\chi(a)\chi(t) = \chi(a)\sum_t\chi(t)$. Thus $\sum_t\chi(t)=0$.
	\item
	if $\chi=\e$, trivially, $\sum_t\chi(t)=p$.
\end{enumerate}
\end{proof}
\begin{proposition}\mbox{}
\begin{enumerate}
	\item
Let $G=\F_p$, then the multiplicative characters of $G$ form a cyclic group of order $p-1$. 
\item
If $a\in\F_p^\times$ and $a\ne1$, then there is a character $\chi$ such that $\chi(a)\ne1$.
\end{enumerate}
\end{proposition}
\begin{definition}
The group of the multiplicative characters of a group $G$ is denoted by $\hat{G}$.
\end{definition}
\begin{proof}\mbox{}
\begin{enumerate}
	\item
It's easy to verify that if $\chi,\lambda$ are characters, then $\chi\lambda$ is also a character. Also, the inverse $\chi^{-1}$ defined by $\chi^{-1}:a\mapsto \chi(a)^{-1}$ is a character. So all characters form a group. 
Let $g$ be a primitive root modulo $p$, then $\xi:=\chi(g)$ is a primitive $p-1_{st}$ root of unity (why? For if exists some a least $u<p-1, u\ne0$ such that $\xi^u = 1$, then $\chi(g^u)=1$. But then exists $v\con{p-1}{u}$ such that $\xi^v=1$, which contradicts our assumption). If there is some $n$ such that $\chi^n=\e$, then $(\chi(g))^{n}=\e(g)=1$. Then $\xi^n=1 \Rightarrow (p-1)\mid n$. But the order of $\xi$ is at most $p-1$, so the order of $\xi$ is $p-1$.
	\item
\notcomplete
\end{enumerate}
\end{proof}
\begin{corollary}
If $a\in\F_p^{\times}, a\ne1,$ then $\sum_{\chi\in\hat{G}}\chi(a)$=0.
\end{corollary}
\begin{proof}
If $a\ne1$, then \begin{align*}
\sum_{\chi\in\hat{G}}\chi(a) &=\sum_{\chi\in\hat{G}}(\mu\chi)(a)\\ 
&=\mu(a)\sum_{\chi\in\hat{G}}\chi(a).
\end{align*}
So $\sum_{\chi\in\hat{G}}\chi(a)=0$.
\end{proof}
We now have the question that \underline{whether $x^n=a$ has a root in $\F_p$?}\\
\begin{proposition}
Let $d = \gcd(n, p-1)$, then $x^n=a$ has a root in $\F_p$ $\Leftrightarrow$ $x^d=a$ has a solution in $\F_p$ and in this case the number of roots is $d$.
\end{proposition}
\begin{proof}
We have proved this result. (Using the fact that $\F_p^\times$ is a cyclic group)
\end{proof}
Another question is that \underline{let $n\mid(p-1)$, does $x^n = a$ has solutions in $\F_p$.}
\begin{theorem}
Let $n\mid p-1, a\in \F_p^\times$, then $$N(x^n=a\mbox{ in }\F_p^\times)=\sum_{\chi\in\hat{G},\chi^n=\e}\chi(a)$$.\end{theorem}
\begin{example}\mbox{}
\begin{enumerate}
	\item $n=2$, $p$ is an odd prime. Then
	$$N(x^2=a\mbox{ in }\F_p^\times)=\sum_{\chi\in\hat{G},\chi^2=\e}\chi(a)=\e(a)+\leg{a}{p}$$
	\item $n=3$, $p$ is an odd prime. Then
	$$N(x^3=a\mbox{ in }\F_p^\times)=\sum_{\chi\in\hat{G},\chi^3=\e}\chi(a)=\e(a)+\chi(a)+\chi^2(a)$$where $\chi$ is a character of order $3$.
\end{enumerate}
\end{example}
We can extend the character $\chi:\F_p^\times\rightarrow\C^\times$ to $\chi:\F_p\rightarrow\C$ by setting $\e(0)=1$ and for $\chi\ne\e$, $\chi(0)=0$.
\begin{proof}
\notcomplete
\end{proof}