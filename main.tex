\documentclass{mynotes}
\usepackage{mymacro}
\begin{document}\tableofcontents
\chapter{Unique Factorization}
\section{Class Notes 17-01-12}
For us, ring means commutative ring with identity.
\begin{definition}
A \emph{ring} is a set with two binary operations $(+,\cdot)$ satisfying
\begin{enumerate}
\item $(R,+)$ is an \emph{abelian group}, which means
\begin{itemize}
\item $+$ is commutative and associative.
\item $\exists\ 0_R, a+0_R=0_R+a$ for all $a\in R$.
\item Given $a\in R$, $\exists a'\in R$ such that $a+a'=0_R$.
\end{itemize}
\item 
$\cdot$ is commutative and associative.\\
$\exists\ 1_R$ such that $a\cdot 1_R=1_R\cdot a =a$ for all $a\in R$.
\item $\cdot$ is distributive over addition, which means
\begin{itemize}
\item $a\cdot(b+c) = a\cdot b+a\cdot c$
\item $(a+b)\cdot c = a\cdot c +b\cdot c$
\end{itemize}
\end{enumerate}
\end{definition}
\begin{exercise}\mbox{}
\begin{enumerate}
\item Show that $a+b=a+c\Rightarrow b=c.$(Cancellation)
\begin{proof}
\begin{align*}
a+b=a+c &\Leftrightarrow a'+ (a+b) = a' + (a+c)\\
&\Leftrightarrow (a'+a) +b = (a'+a) +c\\
&\Leftrightarrow 0_R+b =0_R +c\\
&\Leftrightarrow b=c
\end{align*}
\end{proof}
\item Show $a'$ is unique. We denote this $a'$ by $-a$.
\begin{proof}
if the statement doesn't hold, then there exist $a',a''$ such that $a+a' =0_R=a+a''$. We then apply cancellation and get $a'=a''$.
\end{proof}
\item Show $0_R$ is unique.
\begin{proof}
Say there are two zero element $0_R$ and $0_R'$, then we have
$$0_R =0_R+0_R' = 0_R'$$
\end{proof}
\item Show $1_R$ is unique.
\begin{proof}
Say there are two unit element $1_R$ and $1_R'$, then we have
$$1_R = 1_R\cdot 1_R' =1_R'$$
\end{proof}
\item Show $a\cdot 0_R =0_R \cdot a = 0_R$
\begin{proof}
We know that $a\cdot 0_R +a=a\cdot (0_R+1_R) =a\cdot 1_R =a = 0_R+a$, apply cancellation then we are done.
\end{proof}
\item Show that $(-1_R)\cdot a =-a$.
\begin{proof}
Since $a\cdot 0_R = 0_R$, we have $a\cdot(1_R+(-1_R)) =0_R$ or $a+ (-1_R)\cdot a = 0_R$. Then $-a=(-1_R)\cdot a$, for $a'$ is unique.
\end{proof}
\item The zero ring is the ring with $1$ element. Show $R$ is zero ring $\Leftrightarrow$ $1_R = 0_R$.
\begin{proof}\mbox{}\\
$``\Rightarrow":$ Trivial.\\
$``\Leftarrow":$ Since we have $a\cdot 1_R =1_R\cdot a =a$ for all $a\in R$ and $1_R=0_R$, we have $0_R = a\cdot 0_R= a$ for all $a\in R$.
\end{proof}
\item Does cancellation hold for $\cdot$\;?\\
\textit{Sol.} No. Consider $a\cdot b = a\cdot c$ and $a\neq 0_R$, then $a\cdot(b-c)=0_R$. So if $R$ is an \emph{integral domain}, then we can apply cancellation of non-zero element.
\end{enumerate}
\end{exercise}
\begin{definition}
$R$ is said to be an \emph{integral domain}  if 
$$a\cdot b =0\quad\Longleftrightarrow\quad a=0\ \mbox{ or }\ b=0.$$
\end{definition}
\begin{definition}
$R$ is said to be a field if every non-zero element in $R$ has a multiplication inverse.
\end{definition}
\begin{exercise}\mbox{}
\begin{enumerate}
\item If $R$ is an integral domain, then we can apply cancellation of non-zero element.
\item Show that every field is an integral domain.
\begin{proof}
If $a\cdot b = 0$ and $a\neq 0_R$, let $a'$ be the multiplication inverse of $a$, then $b=1_R \cdot b=a'\cdot a\cdot b =a'\cdot0_R=0$.
\end{proof}
\item Check that $a^{-1}$ is unique.
\begin{proof}
If $a^{-1}$ and $a'$ are both multiplication inverse of $a$, then $a\cdot a^{-1} = a\cdot a' =1_R$. Apply cancellation of non-zero element, we have $a' = a^{-1}$.
\end{proof}
\end{enumerate}
\end{exercise}
\begin{remark}Though every field is an integral domain, not every integral domain is a field. For example, $\mathbb{Z}$ is an integral domain but not a field.
\end{remark}\mbox{}\\
\underline{\bfseries Ways to make new rings:}\\[1pt]
Let $R$ be an integral domain, how to construct a new ring?\\
Let $K=\{(a,b), a,b\in R, b\neq 0\}$. We also define an equivalent relation $(a,b)\sim(c,d)$ if $ad=bc$.
\begin{itemize}
\item Check this is an equivalent class.
\begin{itemize}
\item $(a,b)=(a,b)$
\item if $(a,b)\sim (c,d)$ and $(c,d)\sim (e,f)$, then $(a,b)\sim (e,f)$
\end{itemize}
\item We define \begin{itemize}
 \item $(a,b)+(c, d) = (ad+bc. bd)$
 \item $(a,b)\cdot (c,d) = (ac, bd)$
 \end{itemize}
 Check these two operation pass to equivalent class.
\item $0_K=[(0,1_R)]$, $1_K = [(1_R, 1_R)]$
\end{itemize}
\begin{definition}
If $R,S$ are two rings, a homomorphism $\phi:R\rightarrow S$ is a map such that \begin{enumerate}
\item $\phi(1_R)=1_S$.
\item $\phi(a+b) = \phi(a)+\phi(b)$.
\item $\phi(ab) = \phi(a)\phi(b)$.
\end{enumerate}
An isomorphism is a homomorphism that is both injective and surjective.
\end{definition}
$\phi: R\rightarrow S$, $a\mapsto [(a,1_R)]$ is an injective homomophism. For example, we have $\mathbb{Z}\subset \mathbb{Q}$.
\begin{remark}
If $R$ is a field, then the homomorphism is isomorphism, i.e., $\phi$ is also surjective. Because for any $[(a,b)]\in K$, we have $\phi(ab^{-1})=[(ab^{-1},1)] = [(a,b)]$.
\end{remark}\mbox{}\\
\underline{\bfseries Ways to kill elements:}
\begin{definition}
An ideal $I$ in $R$ is a non-empty subset such that 
\begin{enumerate}
\item $I$ is closed under addition.
\item $I$ is closed under multiplication by arbitrary elt in $R$.
\end{enumerate}
Note that $(I, +)\subset (R,+)$ is an abelian subgroup.
\end{definition}
\begin{example}\mbox{}
\begin{itemize}
\item $(0)$ is an ideal.
\item $R$ itself is an ideal.
\item if $a\in R$, the $R\cdot a$ is an ideal, denoted by $(a)_R$.
\item $n\mathbb{Z}$ is an ideal in $\mathbb{Z}$.
\end{itemize}
\end{example}
\underline{\bfseries Quotient Ring:} Let $I\subset R$ be an ideal. $R/I=$ coset of $I$ in $R$ $=\{a+I,a\in R\}$, we define
\begin{enumerate}
\item $(a+I)\oplus(b+I) = (a+b)+I$.
\item $(a+I)\odot(b+I)=ab+I$.
\end{enumerate}
with zero elt $(0+I)$ and identity elt $(1+I)$.
\section{Unique Factorization in $\mathbb{Z}$}
It will be more convenient to work with $\mathbb{Z}$ rather than restricting ourselves to the positive integers. The notion of divisibility carries over with no difficulty to $\mathbb{Z}$. If $p$ is a positive prime, $-p$ will also be a prime. We shall not consider $1$ or $-1$ as primes even though they fit the definition. This is simply a useful convention. They are called the units of $\mathbb{Z}$.\\There are a number of simple properties of division that we shall simply list.
\begin{enumerate}
\item $a\mid a,a\neq0$.
\item If $a\mid b$ and $b\mid a$, then $a=\pm b$.
\item If $a\mid b$ and $b\mid c$, then $a\mid c$.
\item If $a\mid b$ and $a\mid c$, then $a\mid (b+c)$.
\end{enumerate}
\begin{lemma}
Every nonzero integer can be written as a product of primes.
\end{lemma}
\begin{theorem}
For every nonzero integer $n$ there is a prime factorization$$n=(-1)^{\varepsilon(n)}\prod_pp^{a(p)},$$with the exponents uniquely determined by $n$. In fact, we have $a(p)=\mbox{ord}_pn$.
\end{theorem}
The proof if this theorem if is not as easy as it may seem. We shall postpone the proof until we have established a few preliminary results.
\begin{lemma}
If $a,b\in\mathbb{Z}$ and $b\ge0$, there exist $q,r\in\mathbb{Z}$ such that $a=qb+r$ with $0\leq r<b$.
\end{lemma}
\begin{definition}
If $a_1,a_2,\ldots,a_n\in\mathbb{Z}$, we define $(a_1,a_2,\ldots,a_n)$ to be the set of all integers of the form $a_1x_1+a_2x_2+\cdots+a_nx_n$ with $x_1,x_2,\ldots,x_n\in\mathbb{Z}$.
\end{definition}
\begin{remark}
Let $A = (a_1,a_2,\ldots,a_n)$. Notice that the sum and difference of two elements in $A$ are again in $A$. Also, if $a\in A$ and $r\in\mathbb{Z}$, then $ra\in A$, i.e., $A$ is an ideal in the ring $\mathbb{Z}$
\end{remark}
\begin{lemma}
If $a,b\in \mathbb{Z}$, then there is a $d\in \mathbb{Z}$ such that $(a, b)=(d)$
\end{lemma}
\begin{definition}
Let $a,b\in\mathbb{Z}$. An integer $d$ is called a greatest common divisor of $a$ and $b$ if $d$ is a divisor of both $a$ and $b$ and if every other common divisor of $a$ and $b$ divides $d$.
\end{definition}
\begin{remark}
The gcd of two numbers, if it exists, is determined up to sign.
\end{remark}
\begin{lemma}
Let $a,b\in\mathbb{Z}$. If $(a,b) = (d)$ then $d$ is a greatest common divisor of $a$ and $b$.
\end{lemma}
\begin{definition}
We say that two integers $a$ and $b$ are relatively prime if the only common divisors are $\pm1$, the units.
\end{definition}
It's fairly standard to use the notation $(a, b)$ for the greatest common divisor of $a$ and $b$. With this convention we can say that $a$ and $b$ are relatively prime if $(a,b)=1$.
\begin{proposition}
Suppose that $a\mid bc$ and that $(a,b)=1$. Then $a\mid c$.
\end{proposition}
\begin{corollary}
If $p$ is a prime and $p\mid bc$, then either $p\mid b$ or $p\mid c$.
\end{corollary}
\begin{corollary}
Suppose that $p$ is a prime and that $a,b\in\mathbb{Z}$. Then $\mbox{ord}_pab=\mbox{ord}_pa+\mbox{ord}_pb.$
\end{corollary}
\section{Class Notes 17-01-12}\label{PIDUFD}
\begin{definition}
A non-zero element in \R{} is called a unit if $\exists\, v\in \R$ such that $uv=1_\R$.
\end{definition}
\begin{definition}
Two element $a,b \in\R$ are said to be associative if $\exists\, a\in\R$ such that $a=bu$, denoted by $a\sim b$.
\end{definition}
\begin{definition}
A non-zero element $\pi$ in \R{} is said to be irreducible if $\pi$ is not a unit and if $a\mid \pi\Rightarrow$ $a$ is a unit or $a$ is associative of $\pi$.
\end{definition}
\begin{definition}
A non-zero element in $\R$ is said to be prime if $\pi$ is not a unit and $\pi\mid  ab\Rightarrow \pi\mid a$ or $\pi\mid b$, $\forall a,b\in\R.$
\end{definition}
\begin{proposition}
If $\pi$ is a prime, then $\pi$ is irreducible.
\end{proposition}
\begin{proof}
Let $\pi$ be a prime, suppose $a\mid \pi$, then $\pi=ab$ for some $b\in\R$. Thus $\pi\mid ab$ and by definition, $\pi\mid a$ or $\pi\mid b$.
\begin{itemize}
\item If $\pi\mid a$, then $a\sim \pi$.
\item If $\pi\mid b$, then $a\sim 1$.
\end{itemize}
\end{proof}
\begin{remark}
A irreducible is not necessary to be a prime.\\
Let $R = \Z[\sqrt{5}] = \{a+b\sqrt{-5}\ \mid \ a,b\in\Z\}\subset \C$. We have $$6=2\cdot3 = (1+\sqrt{-5})\cdot(1-\sqrt{-5}).$$We write $\pi = (1+\sqrt{-5})$ and claim that $2,3,\pi,\overline{\pi}$ are irreducibles but none of them are associative of each other.\\
We define the norm function $N:R\rightarrow \Z$, where $N(\al)=\al\overline{\al}$, i.e., if $\al = a+bi$, then $N(\al)=a^2+5b^2$. We notice that 
\begin{itemize}
\item If $\al>0$, then $N(\al)>0$.
\item $N(\al\be)=N(\al)N(\be)$.
\end{itemize}
\underline{Check: $2$ is irreducible:}\\
\underline{Find unit:} \\$N(uv) = N(1) = 1 = N(u)N(v) \Rightarrow N(u)=N(v) = 1$. But $a^2+5b^2 = 1\Rightarrow a=\pm1, b=0$.\\Suppose $2=\al\be$, then $4=N(2) = N(\al\be) = N(\al)N(\be)$.
\begin{enumerate}
\item If $N(\al) = 1, N(\be) = 4$\\
Then $\al$ is a unit $\Rightarrow$ $2$ is irreducible.
\item If $N(\al) = 2, N(\be) = 2$\\
Then $a^2+5b^2 = 2$ has no solution.
\end{enumerate}
\end{remark}
\begin{definition}
An UFD (Unique Factorization Domain) is an integral domain $R$ in which every non-zero element(up to unit) factors uniquely into a product of irreducibles.
\end{definition}
\begin{proposition}
Let $R$ be a domain in which factorization (of irreducibles) exists. Then \emph{$R$ is a UFD $\Leftrightarrow$ every irreducible in $R$ is prime.}
\end{proposition}
\begin{proof}\mbox{}\\
$``\Leftarrow":$ Let $a$ be an element of $R$ and $a\ne0$. If $a=\pi_1\pi_2\cdots\pi_n=\sigma_1\sigma_2\cdots\sigma_m$ are two factorizations. Since $\pi_1$ is prime, $\pi_1\mid \sigma_i$ for some $i$. By rearranging, we may assume $\pi_1\mid \sigma_1$, Thus $\pi_1\sim \sigma_1$. Repeating this process, we can conclude that the two factorizations are the same.
\notcomplete
\end{proof}
\begin{remark}
There are clearly rings such that no factorization exists. For example, consider the ring $\Z[2^{1/2},2^{1/4},2^{1/8},\ldots]\subset \R$. It's the smallest subring of $\R$ that contains $2^{1/2},2^{1/4},\ldots$.
\end{remark}
\begin{definition}
A ring $R$ is said to be noetherian if it satisfies any of the following equivalent conditions:
\begin{enumerate}
\item Any ascending chain of ideals in $R$ terminates.\\
Namely, $I_1\subset I_2\subset I_3\subset \cdots\Rightarrow I_n = I_{n+1}=\cdots$ for some $n$.
\item Any ideal $I$ in $R$ is finite generated.\\
Namely, $I=(a_1,\ldots,a_n)$ for some $n$.
\end{enumerate}
\end{definition}
\begin{proof}
\mbox{}\\\mbox{``1.\ $\Rightarrow$2.\ ": } Let $I$ be an ideal, if $I\ne0$, pick $a_1\in I, a_1\neq 0$, clearly $(a_1)\subset I$. 
If $(a_1)=I$, we are done, If not,
$\exists a_2\in I\backslash(a_1)\Rightarrow (a_1,a_2)\subset I $, this chain terminates.\\
\mbox{``1.\ $\Leftarrow$2.\ ": }Suppose $I_1\subset I_2\subset \ldots$ be an ascending ideal. Let $I = \cup I_n$, we claim that $I$ is an ideal.\\
Let $a,b\in I$, then there exists $n$ such that $a,b\in I_n$. Therefore $a+b\in I_n$, and $a+b\in I$. Let $a\in I$, then $a\in I_n$ for some $n$. Therefore $ra\in I_n\implies ra\in I$. Thus $I$ is an ideal. But $I=(a_1,\ldots,a_m)$, so there exists $n$, such that $a_1,\ldots,a_m\in I_n$. Thus $I=I_n$ and $I_n=I_{n+1}=\cdots$.
\end{proof}
\begin{exercise}
Suppose $R$ is a Noetherian domain, show $R$ admits factorizations.
\end{exercise}
\begin{proof}
If $b$ is not irreducible, then $b=ac$ or $(b)\subset(a)$\notcomplete
\end{proof}
\begin{definition}
A PID (Principle Ideal Domain) is a domain in which every ideal is generated by a single element.
\end{definition}
\begin{theorem}
Every PID is a UFD.
\end{theorem}
\begin{proof}
Let $R$ be a PID, then it's noetherian. So factorizations exist. So it suffices to show that every irreducible is a prime. Let $\pi$ be a irreducible in $R$. Suppose $\pi\mid ab$ and $a$ is not divided by $\pi$. We look at $I=(a,\pi)$, there exists $c\in R$, such that $I=(c)$. Thus we have $c\mid \pi, c\mid a$. So $c\sim 1$ or $c\sim \pi$. Since $c$ is not associative of $\pi$, $c$ is associative of $1$. But then $$1=ax+\pi y$$ for some $x,y\in R$. So $b=abx+\pi by$ or $\pi\mid b$.
\end{proof}
\section{Class Notes 17-01-17}
\begin{example}
\Z{} is a PID.
\end{example}
\begin{remark}
Any ideal $I\subset\Z$ is of the form of $n\Z$.
\end{remark}
\begin{proof}
$\forall I\subset \Z$, if $I=(0)$, we are done. If $I$ is not zero ideal, let $n$ be the smallest positive element in $I$. We claim: $I=n\Z$. Let $b\in I$, then $b=nq+r$, where $0\leq r<n$. But $r=b-nq\implies r\in I\implies r=0$. Therefore $b= nq$. 
\end{proof}
If $K$ is a field, let $R = k[x] = \mbox{polynomial in variable $x$ over the field $K$}$. What are the units in $R$? For arbitrary $f(x),g(x)\in K[x]$, if $f(x)g(x)=1$, we claim that $f(x), g(x)$ must be constant polynomial. For if we write $f(x)=a_nx^n+a_{n-1}x^{n-1}+\cdots$, $g(x)=b_mx^m+b_{m-1}x^{m-1}+\cdots$. Then $f(x)g(x) = a_nb_m x^{m+n}+\cdots$. Since $a_n\ne0,b_m\ne0$ and \emph{$K$ is an integral domain}, we have $a_nb_m\ne0$. Therefore $$\deg{f(x)g(x)} = \deg{f(x)}+\deg{g(x)}.$$ We then apply this conclusion to $f(x)g(x) = 1$ and get $\deg{f(x)}\deg{g(x)} = \deg{1} = 0$, thus $f(x), g(x)$ must be constant.
\begin{remark}
Whether a polynomial is irreducible depends on the field. For example, if $x^2+1\in \R[x]$, then it's irreducible (why?). But if $x^2+1\in \C[x]$, then it's reducible (why?).
\end{remark}
\underline{Division Algorithm:} Let $f(x), g(x)\in K[x], g(x)\ne 0$, then there exists $q(x),r(x)\in K[x]$, such that $$f(x)=g(x)q(x)+r(x),$$where $r(x)=0$ or $0\leq\deg{r(x)}<\deg{g(x)}$. \\
Using this fact, we have the following theorem.
\begin{theorem}\label{kxpid}
$K[x]$ is a PID.
\end{theorem}
\begin{proof}
For all ideal $I\in K[x]$, if $I=(0)$, we are done. If $I\ne(0)$, let $g(x)\in I$ be the polynomial of least degree, let $f(x)\in I$, then $$f(x)=g(x)q(x)+r$$with $r=0$ or $0\leq\deg{r(x)}<\deg{g(x)}$ by division algorithm. But then $r(x)=0$, for otherwise $r(x)$ will be a polynomial whose degree is less than $g(x)$. Therefore $f(x)=g(x)q(x)$, $f(x)\in (g(x))$.
\end{proof}
\begin{definition}
A domain $R$ is said to be an Euclidean domain if there exists a function $\lambda: \R\setminus\{0\}\rightarrow\Z^{\ge0}$, such that given $a,b\in R, b\ne0$, there exist $q,r\in R$ such that $a=qb+r$ and either $r=0$ or $0\le\lambda(r)<\lambda(b)$.
\end{definition}
\begin{example}
$R=\Z[i]$ is an Euclidean domain.
\end{example}
\begin{proof}
Let $N(\al)=\al\overline{\al} = a^2+b^2$ (if $\al = a+bi$). Let $\al,\be\in R$, $\be\ne0$, we have
$$\frac{\al}{\be} = \frac{a+bi}{c+di} = \frac{ac+bd}{c^2+d^2}+\frac{bc-ad}{c^2+d^2}i = r+si. (r,s\in\Q)$$Let $m+ni\in\Z[i]$ be the closest element to $r+si$. We denote $r'=r-m,s'=s-n$, then $\frac{\al}{\be} = r+si = m+ni+r'+s'i$, or 
$$\al = \be(m+ni)+\be(r'+s'i),$$where $(m+ni)\in\Z[i]$ and $\be(r'+s'i)\in\Z[i]$, we remain to show that $N(\be(r'+s'i))<N(\be)$. This is the case because 
\begin{align*}
N(\be(r'+s'i))&= N(\be)N(r'+s'i)\\&\leq N(\be)(\frac{1}{4}+\frac{1}{4})\\&<N(\be)
\end{align*} We are done.
\end{proof}
The Natural question is what are the units in $\Z[i]$? Does a prime in \Z{} still a prime in $Z[i]$? To answer the first question, we assume $u$ is a unit in $\Z[i]$. Then by definition there exists some $v$ such that $uv= 1$. But then $1=N(1) = N(uv) = N(u)N(v)\implies N(u) = 1$. Thus the only possible values of $u$ is $\pm1,\pm i$. We also check they are actually units. Now, to answer the second question, we try some small cases. We look at $5,7,11$ and $13$.
\begin{example}
If $5=ab$, $a,b\in \Z[i]$, then $25=N(5) = N(ab) = N(a)N(b)\implies N(a) = 5$. So $a$ can only be $\pm1\pm2i$ or $\pm2\pm i$. We try by hand and find $5= (2+i)(2-i)$ is a factorization, so $5$ is not a prime. 
\end{example} 
\begin{example}
If $7=ab$, $a,b\in \Z[i]$, then $49=N(5) = N(ab) = N(a)N(b)\implies N(a) = 7$. We try by hand and find no factorization, so $7$ is a prime.
\end{example}
Use the same method, we find $5,13$ are not prime while $7,11$ are prime.
\begin{remark}\mbox{}Obervation:\\
\begin{enumerate}
	\item If $p\con{1}{4}$, then $p = \pi\overline{\pi}$, where $\pi$ is a irreducible.
	\item If $p\con{3}{4}$, then $p$ remains prime.
	\item If $p=2$, $2=(1+i)(1-i) = (-i)(1+i)^2$ (ramification).
\end{enumerate}
\end{remark}
\begin{remark}
Let $R = \Z[\omega]$, where $\omega$ is a primitive cube root of $1$, then $R$ is a Euclidean domain.
\end{remark}
\section{Unique Factorization in $k[x]$}
In this section we consier the ring $k[x]$ of polynomials with coefficients in a field $k$. If $f,g\in k[x]$, we say that $f$ divides $g$ if there is an $h\in k[x]$ such that $g= fh$.\\
If $\deg{f}$ denotes the degree of $f$, we have $\deg{fg}= \deg{f}+\deg{g}$ (why? Because a field $k$ is necessarily an integral domain). nonzeros constants are the units of $k[x]$. A nonconstant polynomial $p$ is said to be irreducible if $q\mid p\implies$ $q$ is either a constant or a constant times $p$.
\begin{lemma}\label{l1}
Every nonconstant polynomial is the product of irreducible polynomials.
\end{lemma}
\begin{proof}
Simply by induction.
\end{proof}
\begin{definition}
A polynomial $f$ is called monic if its leading coefficient is $1$.
\end{definition}
\begin{definition}
Let $p$ be a monic irreducibe polynomial. We define $\ord{p}{f}$ to be the integer $a$ defined by the property that $p^a\mid f$ but that $p^{a+1}\nmid f$.
\end{definition}
\begin{remark}
$\ord{p}{f} = 0$ iff $p\nmid f$.
\end{remark}
\begin{theorem}
Let $f\in k[x]$. Then we can write$$f= c\prod_pp^{a(p)},$$ where the product is over all monic irreducible polynomials and $c$ is a constant. The constant $c$ and the exponents $a(p)$ are uniquely determined by $f$; in fact, $a(p)= \ord{p}{f}$.
\end{theorem}
The existence of such a product follows immediately from Lemma \ref{l1}. The uniqueness part is more difficult and will be postponed.
\begin{lemma}
Let $f,g\in k[x]$. If $g\ne 0$, there exist polynomials $h,r\in k[x]$ such that $f=hg+r$, where either $r=0$ or $r\ne0$ and $\deg{r}\le \deg{g}$.
\end{lemma}
\begin{proof}
If $g\mid f$, we are done. If $g\nmid f$, let $r= f-hg$ be the polynomial of least degree among all polynomials of the form $f-lg$ with $l\in k[x]$. We claim that $\deg{r}<\deg{g}$. If not, let the leading term of $r$ be $ax^d$ and that $g$ be $bx^m$. Then $r-\frac{a}{b}x^{d-m}g(x)= f-(h+\frac{a}{b}x^{d-m})g$ has smaller degree than $r$ and is of the given form. This is a contradiction.
\end{proof}
\begin{lemma}\label{gcdfgd}
Given $f,g\in k[x]$ there is a $d\in k[x]$ such that $(f,g)=(d)$.
\end{lemma}
\begin{proof}
See Theorem \ref{kxpid}.
\end{proof}
\begin{definition}
Let $f,g\in k[x]$. Then $d\in k[x]$ is said to be a greatest common divisor of $f$ and $g$ if $d$ divides $f$ and $g$ and every common divisor of $f$ and $g$ divides $d$.
\end{definition}
\begin{remark}
Notice that the greatest common divisor of two polynomials is determined up to multiplication by a constant. If we require it to be monic, it is uniquely determined and we may speak of the greatest common divisor.
\end{remark}
\begin{lemma}
Let $f,g\in k[x]$ By lemma \ref{gcdfgd} there is a $d\in k[x]$ such that $(f,g)=(d)$. $d$ is the greatest common divisor of $f$ and $g$.
\end{lemma}
\begin{proof}
Since $f\in (d)$ and $g\in(d)$ we have $d\mid f$ and $d\mid g$. Suppose that $h\mid f$ and that $h\mid g$. Then $h$ divides every elements in $(f,g)=(d)$. In particular $h\mid d$, we are done.
\end{proof}
\begin{definition}
Two polynomial $f$ and $g$ are said to be relatively prime if the only common divisor of $f$ and $g$ are constants. In other words, $(f,g) = (1)$.
\end{definition}
\begin{proposition}
If $f$ and $g$ are relatively prime and $f\mid gh$, then $f\mid h$.
\end{proposition}
\begin{corollary}
If $p$ is an irreducible polynomial and $p\mid fg$, then $p\mid g$ or $p\mid g$. 
\end{corollary}
\begin{corollary}
If $p$ is a monic irreducible polynomial and $f,g\in k[x]$, we have $$\ord{p}{fg}=\ord{p}{f}+\ord{p}{g}.$$
\end{corollary}
Using these tools, we can prove the uniqueness of factorizaion.
\section{Unique Factorizaion in a Principal Ideal Domain}
For this section, we mostly refer to Section \ref{PIDUFD} and supply some details. 
\end{document}
































