\documentclass{mynotes}
\usepackage{mymacro}
\begin{document}\tableofcontents
\chapter{Unique Factorization}
For us, ring means commutative ring with identity.
\begin{definition}
A \emph{ring} is a set with two binary operations $(+,\cdot)$ satisfying
\begin{enumerate}
\item $(R,+)$ is an \emph{abelian group}, which means
\begin{itemize}
\item $+$ is commutative and associative.
\item $\exists\ 0_R, a+0_R=0_R+a$ for all $a\in R$.
\item Given $a\in R$, $\exists a'\in R$ such that $a+a'=0_R$.
\end{itemize}
\item 
$\cdot$ is commutative and associative.\\
$\exists\ 1_R$ such that $a\cdot 1_R=1_R\cdot a =a$ for all $a\in R$.
\item $\cdot$ is distributive over addition, which means
\begin{itemize}
\item $a\cdot(b+c) = a\cdot b+a\cdot c$
\item $(a+b)\cdot c = a\cdot c +b\cdot c$
\end{itemize}
\end{enumerate}
\end{definition}
\begin{exercise}\mbox{}
\begin{enumerate}
\item Show that $a+b=a+c\Rightarrow b=c.$(Cancellation)
\begin{proof}
\begin{align*}
a+b=a+c &\Leftrightarrow a'+ (a+b) = a' + (a+c)\\
&\Leftrightarrow (a'+a) +b = (a'+a) +c\\
&\Leftrightarrow 0_R+b =0_R +c\\
&\Leftrightarrow b=c
\end{align*}
\end{proof}
\item Show $a'$ is unique. We denote this $a'$ by $-a$.
\begin{proof}
if the statement doesn't hold, then there exist $a',a''$ such that $a+a' =0_R=a+a''$. We then apply cancellation and get $a'=a''$.
\end{proof}
\item Show $0_R$ is unique.
\begin{proof}
Say there are two zero element $0_R$ and $0_R'$, then we have
$$0_R =0_R+0_R' = 0_R'$$
\end{proof}
\item Show $1_R$ is unique.
\begin{proof}
Say there are two unit element $1_R$ and $1_R'$, then we have
$$1_R = 1_R\cdot 1_R' =1_R'$$
\end{proof}
\item Show $a\cdot 0_R =0_R \cdot a = 0_R$
\begin{proof}
We know that $a\cdot 0_R +a=a\cdot (0_R+1_R) =a\cdot 1_R =a = 0_R+a$, apply cancellation then we are done.
\end{proof}
\item Show that $(-1_R)\cdot a =-a$.
\begin{proof}
Since $a\cdot 0_R = 0_R$, we have $a\cdot(1_R+(-1_R)) =0_R$ or $a+ (-1_R)\cdot a = 0_R$. Then $-a=(-1_R)\cdot a$, for $a'$ is unique.
\end{proof}
\item The zero ring is the ring with $1$ element. Show $R$ is zero ring $\Leftrightarrow$ $1_R = 0_R$.
\begin{proof}\mbox{}\\
$``\Rightarrow":$ Trivial.\\
$``\Leftarrow":$ Since we have $a\cdot 1_R =1_R\cdot a =a$ for all $a\in R$ and $1_R=0_R$, we have $0_R = a\cdot 0_R= a$ for all $a\in R$.
\end{proof}
\item Does cancellation hold for $\cdot$\;?\\
\textit{Sol.} No. Consider $a\cdot b = a\cdot c$ and $a\neq 0_R$, then $a\cdot(b-c)=0_R$. So if $R$ is an \emph{integral domain}, then we can apply cancellation of non-zero element.
\end{enumerate}
\end{exercise}
\begin{definition}
$R$ is said to be an \emph{integral domain}  if 
$$a\cdot b =0\quad\Longleftrightarrow\quad a=0\ \mbox{ or }\ b=0.$$
\end{definition}
\begin{definition}
$R$ is said to be a field if every non-zero element in $R$ has a multiplication inverse.
\end{definition}
\begin{exercise}\mbox{}
\begin{enumerate}
\item If $R$ is an integral domain, then we can apply cancellation of non-zero element.
\item Show that every field is an integral domain.
\begin{proof}
If $a\cdot b = 0$ and $a\neq 0_R$, let $a'$ be the multiplication inverse of $a$, then $b=1_R \cdot b=a'\cdot a\cdot b =a'\cdot0_R=0$.
\end{proof}
\item Check that $a^{-1}$ is unique.
\begin{proof}
If $a^{-1}$ and $a'$ are both multiplication inverse of $a$, then $a\cdot a^{-1} = a\cdot a' =1_R$. Apply cancellation of non-zero element, we have $a' = a^{-1}$.
\end{proof}
\end{enumerate}
\end{exercise}
\begin{remark}Though every field is an integral domain, not every integral domain is a field. For example, $\mathbb{Z}$ is an integral domain but not a field.
\end{remark}\mbox{}\\
\underline{\bfseries Ways to make new rings:}\\[1pt]
Let $R$ be an integral domain, how to construct a new ring?\\
Let $K=\{(a,b), a,b\in R, b\neq 0\}$. We also define an equivalent relation $(a,b)\sim(c,d)$ if $ad=bc$.
\begin{itemize}
\item Check this is an equivalent class.
\begin{itemize}
\item $(a,b)=(a,b)$
\item if $(a,b)\sim (c,d)$ and $(c,d)\sim (e,f)$, then $(a,b)\sim (e,f)$
\end{itemize}
\item We define \begin{itemize}
 \item $(a,b)+(c, d) = (ad+bc. bd)$
 \item $(a,b)\cdot (c,d) = (ac, bd)$
 \end{itemize}
 Check these two operation pass to equivalent class.
\item $0_K=[(0,1_R)]$, $1_K = [(1_R, 1_R)]$
\end{itemize}
\begin{definition}
If $R,S$ are two rings, a homomorphism $\phi:R\rightarrow S$ is a map such that \begin{enumerate}
\item $\phi(1_R)=1_S$.
\item $\phi(a+b) = \phi(a)+\phi(b)$.
\item $\phi(ab) = \phi(a)\phi(b)$.
\end{enumerate}
An isomorphism is a homomorphism that is both injective and surjective.
\end{definition}
$\phi: R\rightarrow S$, $a\mapsto [(a,1_R)]$ is an injective homomophism. For example, we have $\mathbb{Z}\subset \mathbb{Q}$.
\begin{remark}
If $R$ is a field, then the homomorphism is isomorphism, i.e., $\phi$ is also surjective. Because for any $[(a,b)]\in K$, we have $\phi(ab^{-1})=[(ab^{-1},1)] = [(a,b)]$.
\end{remark}\mbox{}\\
\underline{\bfseries Ways to kill elements:}
\begin{definition}
An ideal $I$ in $R$ is a non-empty subset such that 
\begin{enumerate}
\item $I$ is closed under addition.
\item $I$ is closed under multiplication by arbitrary elt in $R$.
\end{enumerate}
Note that $(I, +)\subset (R,+)$ is an abelian subgroup.
\end{definition}
\begin{example}\mbox{}
\begin{itemize}
\item $(0)$ is an ideal.
\item $R$ itself is an ideal.
\item if $a\in R$, the $R\cdot a$ is an ideal, denoted by $(a)_R$.
\item $n\mathbb{Z}$ is an ideal in $\mathbb{Z}$.
\end{itemize}
\end{example}
\underline{\bfseries Quotient Ring:} Let $I\subset R$ be an ideal. $R/I=$ coset of $I$ in $R$ $=\{a+I,a\in R\}$, we define
\begin{enumerate}
\item $(a+I)\oplus(b+I) = (a+b)+I$.
\item $(a+I)\odot(b+I)=ab+I$.
\end{enumerate}
with zero elt $(0+I)$ and identity elt $(1+I)$.
\section{Unique Factorization in $\mathbb{Z}$}
It will be more convenient to work with $\mathbb{Z}$ rather than restricting ourselves to the positive integers. The notion of divisibility carries over with no difficulty to $\mathbb{Z}$. If $p$ is a positive prime, $-p$ will also be a prime. We shall not consider $1$ or $-1$ as primes even though they fit the definition. This is simply a useful convention. They are called the units of $\mathbb{Z}$.\\There are a number of simple properties of division that we shall simply list.
\begin{enumerate}
\item $a|a,a\neq0$.
\item If $a|b$ and $b|a$, then $a=\pm b$.
\item If $a|b$ and $b|c$, then $a|c$.
\item If $a|b$ and $a|c$, then $a|(b+c)$.
\end{enumerate}
\begin{lemma}
Every nonzero integer can be written as a product of primes.
\end{lemma}
\begin{theorem}
For every nonzero integer $n$ there is a prime factorization$$n=(-1)^{\varepsilon(n)}\prod_pp^{a(p)},$$with the exponents uniquely determined by $n$. In fact, we have $a(p)=\mbox{ord}_pn$.
\end{theorem}
The proof if this theorem if is not as easy as it may seem. We shall postpone the proof until we have established a few preliminary results.
\begin{lemma}
If $a,b\in\mathbb{Z}$ and $b\ge0$, there exist $q,r\in\mathbb{Z}$ such that $a=qb+r$ with $0\leq r<b$.
\end{lemma}
\begin{definition}
If $a_1,a_2,\ldots,a_n\in\mathbb{Z}$, we define $(a_1,a_2,\ldots,a_n)$ to be the set of all integers of the form $a_1x_1+a_2x_2+\cdots+a_nx_n$ with $x_1,x_2,\ldots,x_n\in\mathbb{Z}$.
\end{definition}
\begin{remark}
Let $A = (a_1,a_2,\ldots,a_n)$. Notice that the sum and difference of two elements in $A$ are again in $A$. Also, if $a\in A$ and $r\in\mathbb{Z}$, then $ra\in A$, i.e., $A$ is an ideal in the ring $\mathbb{Z}$
\end{remark}
\begin{lemma}
If $a,b\in \mathbb{Z}$, then there is a $d\in \mathbb{Z}$ such that $(a, b)=(d)$
\end{lemma}
\begin{definition}
Let $a,b\in\mathbb{Z}$. An integer $d$ is called a greatest common divisor of $a$ and $b$ if $d$ is a divisor of both $a$ and $b$ and if every other common divisor of $a$ and $b$ divides $d$.
\end{definition}
\begin{remark}
The gcd of two numbers, if it exists, is determined up to sign.
\end{remark}
\begin{lemma}
Let $a,b\in\mathbb{Z}$. If $(a,b) = (d)$ then $d$ is a greatest common divisor of $a$ and $b$.
\end{lemma}
\begin{definition}
We say that two integers $a$ and $b$ are relatively prime if the only common divisors are $\pm1$, the units.
\end{definition}
It's fairly standard to use the notation $(a, b)$ for the greatest common divisor of $a$ and $b$. With this convention we can say that $a$ and $b$ are relatively prime if $(a,b)=1$.
\begin{proposition}
Suppose that $a|bc$ and that $(a,b)=1$. Then $a|c$.
\end{proposition}
\begin{corollary}
If $p$ is a prime and $p|bc$, then either $p|b$ or $p|c$.
\end{corollary}
\begin{corollary}
Suppose that $p$ is a prime and that $a,b\in\mathbb{Z}$. Then $\mbox{ord}_pab=\mbox{ord}_pa+\mbox{ord}_pb.$
\end{corollary}
\end{document}
