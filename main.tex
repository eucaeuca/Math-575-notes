\documentclass{mynotes}
\usepackage{mymacro}
\usepackage{diagrams}
\begin{document}\tableofcontents
\chapter{Unique Factorization}
\section{Class Notes 17-01-12}
For us, ring means commutative ring with identity.
\begin{definition}
A \emph{ring} is a set with two binary operations $(+,\cdot)$ satisfying
\begin{enumerate}
\item $(R,+)$ is an \emph{abelian group}, which means
\begin{itemize}
\item $+$ is commutative and associative.
\item $\exists\ 0_R, a=a+0_R=0_R+a$ for all $a\in R$.
\item Given $a\in R$, $\exists a'\in R$ such that $a+a'=0_R$.
\end{itemize}
\item 
$\cdot$ is commutative and associative.\\
$\exists\ 1_R$ such that $a\cdot 1_R=1_R\cdot a =a$ for all $a\in R$.
\item $\cdot$ is distributive over addition, which means
\begin{itemize}
\item $a\cdot(b+c) = a\cdot b+a\cdot c$
\item $(a+b)\cdot c = a\cdot c +b\cdot c$
\end{itemize}
\end{enumerate}
\end{definition}
\begin{exercise}\mbox{}
\begin{enumerate}
\item Show that $a+b=a+c\Rightarrow b=c.$(Cancellation)
\begin{proof}
\begin{align*}
a+b=a+c &\Leftrightarrow a'+ (a+b) = a' + (a+c)\\
&\Leftrightarrow (a'+a) +b = (a'+a) +c\\
&\Leftrightarrow 0_R+b =0_R +c\\
&\Leftrightarrow b=c
\end{align*}
\end{proof}
\item Show $a'$ is unique. We denote this $a'$ by $-a$.
\begin{proof}
if the statement doesn't hold, then there exist $a',a''$ such that $a+a' =0_R=a+a''$. We then apply cancellation and get $a'=a''$.
\end{proof}
\item Show $0_R$ is unique.
\begin{proof}
Say there are two zero element $0_R$ and $0_R'$, then we have
$$0_R =0_R+0_R' = 0_R'$$
\end{proof}
\item Show $1_R$ is unique.
\begin{proof}
Say there are two unit element $1_R$ and $1_R'$, then we have
$$1_R = 1_R\cdot 1_R' =1_R'$$
\end{proof}
\item Show $a\cdot 0_R =0_R \cdot a = 0_R$
\begin{proof}
We know that $a\cdot 0_R +a=a\cdot (0_R+1_R) =a\cdot 1_R =a = 0_R+a$, apply cancellation then we are done.
\end{proof}
\item Show that $(-1_R)\cdot a =-a$.
\begin{proof}
Since $a\cdot 0_R = 0_R$, we have $a\cdot(1_R+(-1_R)) =0_R$ or $a+ (-1_R)\cdot a = 0_R$. Then $-a=(-1_R)\cdot a$, for $a'$ is unique.
\end{proof}
\item The zero ring is the ring with $1$ element. Show $R$ is zero ring $\Leftrightarrow$ $1_R = 0_R$.
\begin{proof}\mbox{}\\
$``\Rightarrow":$ Trivial.\\
$``\Leftarrow":$ Since we have $a\cdot 1_R =1_R\cdot a =a$ for all $a\in R$ and $1_R=0_R$, we have $0_R = a\cdot 0_R= a$ for all $a\in R$.
\end{proof}
\item Does cancellation hold for $\cdot$\;?\\
\textit{Sol.} No. Consider $a\cdot b = a\cdot c$ and $a\neq 0_R$, then $a\cdot(b-c)=0_R$. So if $R$ is an \emph{integral domain}, then we can apply cancellation of non-zero element.
\end{enumerate}
\end{exercise}
\begin{definition}
$R$ is said to be an \emph{integral domain}  if 
$$a\cdot b =0\quad\Longleftrightarrow\quad a=0\ \mbox{ or }\ b=0.$$
\end{definition}
\begin{definition}
$R$ is said to be a field if every non-zero element in $R$ has a multiplication inverse.
\end{definition}
\begin{exercise}\mbox{}
\begin{enumerate}
\item If $R$ is an integral domain, then we can apply cancellation of non-zero element.
\item Show that every field is an integral domain.
\begin{proof}
If $a\cdot b = 0$ and $a\neq 0_R$, let $a'$ be the multiplication inverse of $a$, then $b=1_R \cdot b=a'\cdot a\cdot b =a'\cdot0_R=0$.
\end{proof}
\item Check that $a^{-1}$ is unique.
\begin{proof}
If $a^{-1}$ and $a'$ are both multiplication inverse of $a$, then $a\cdot a^{-1} = a\cdot a' =1_R$. Apply cancellation of non-zero element, we have $a' = a^{-1}$.
\end{proof}
\end{enumerate}
\end{exercise}
\begin{remark}Though every field is an integral domain, not every integral domain is a field. For example, $\mathbb{Z}$ is an integral domain but not a field.
\end{remark}\mbox{}\\
\underline{\bfseries Ways to make new rings:}\\[1pt]
Let $R$ be an integral domain, how to construct a new ring?\\
Let $K=\{(a,b), a,b\in R, b\neq 0\}$. We also define an equivalent relation $(a,b)\sim(c,d)$ if $ad=bc$.
\begin{itemize}
\item Check this is an equivalent class.
\begin{itemize}
\item $(a,b)=(a,b)$
\item if $(a,b)\sim (c,d)$ and $(c,d)\sim (e,f)$, then $(a,b)\sim (e,f)$
\end{itemize}
\item We define \begin{itemize}
 \item $(a,b)+(c, d) = (ad+bc. bd)$
 \item $(a,b)\cdot (c,d) = (ac, bd)$
 \end{itemize}
 Check these two operation pass to equivalent class.
\item $0_K=[(0,1_R)]$, $1_K = [(1_R, 1_R)]$
\end{itemize}
\begin{definition}
If $R,S$ are two rings, a homomorphism $\phi:R\rightarrow S$ is a map such that \begin{enumerate}
\item $\phi(1_R)=1_S$.
\item $\phi(a+b) = \phi(a)+\phi(b)$.
\item $\phi(ab) = \phi(a)\phi(b)$.
\end{enumerate}
An isomorphism is a homomorphism that is both injective and surjective.
\end{definition}
$\phi: R\rightarrow S$, $a\mapsto [(a,1_R)]$ is an injective homomophism. For example, we have $\mathbb{Z}\subset \mathbb{Q}$.
\begin{remark}
If $R$ is a field, then the homomorphism is isomorphism, i.e., $\phi$ is also surjective. Because for any $[(a,b)]\in K$, we have $\phi(ab^{-1})=[(ab^{-1},1)] = [(a,b)]$.
\end{remark}\mbox{}\\
\underline{\bfseries Ways to kill elements:}
\begin{definition}
An ideal $I$ in $R$ is a non-empty subset such that 
\begin{enumerate}
\item $I$ is closed under addition.
\item $I$ is closed under multiplication by arbitrary elt in $R$.
\end{enumerate}
Note that $(I, +)\subset (R,+)$ is an abelian subgroup.
\end{definition}
\begin{example}\mbox{}
\begin{itemize}
\item $(0)$ is an ideal.
\item $R$ itself is an ideal.
\item if $a\in R$, the $R\cdot a$ is an ideal, denoted by $(a)_R$.
\item $n\mathbb{Z}$ is an ideal in $\mathbb{Z}$.
\end{itemize}
\end{example}
\underline{\bfseries Quotient Ring:} Let $I\subset R$ be an ideal. $R/I=$ coset of $I$ in $R$ $=\{a+I,a\in R\}$, we define
\begin{enumerate}
\item $(a+I)\oplus(b+I) = (a+b)+I$.
\item $(a+I)\odot(b+I)=ab+I$.
\end{enumerate}
with zero elt $(0+I)$ and identity elt $(1+I)$.
\section{Unique Factorization in $\mathbb{Z}$}
It will be more convenient to work with $\mathbb{Z}$ rather than restricting ourselves to the positive integers. The notion of divisibility carries over with no difficulty to $\mathbb{Z}$. If $p$ is a positive prime, $-p$ will also be a prime. We shall not consider $1$ or $-1$ as primes even though they fit the definition. This is simply a useful convention. They are called the units of $\mathbb{Z}$.\\There are a number of simple properties of division that we shall simply list.
\begin{enumerate}
\item $a\mid a,a\neq0$.
\item If $a\mid b$ and $b\mid a$, then $a=\pm b$.
\item If $a\mid b$ and $b\mid c$, then $a\mid c$.
\item If $a\mid b$ and $a\mid c$, then $a\mid (b+c)$.
\end{enumerate}
\begin{lemma}
Every nonzero integer can be written as a product of primes.
\end{lemma}
\begin{theorem}
For every nonzero integer $n$ there is a prime factorization$$n=(-1)^{\varepsilon(n)}\prod_pp^{a(p)},$$with the exponents uniquely determined by $n$. In fact, we have $a(p)=\mbox{ord}_pn$.
\end{theorem}
The proof if this theorem if is not as easy as it may seem. We shall postpone the proof until we have established a few preliminary results.
\begin{lemma}
If $a,b\in\mathbb{Z}$ and $b\ge0$, there exist $q,r\in\mathbb{Z}$ such that $a=qb+r$ with $0\leq r<b$.
\end{lemma}
\begin{definition}
If $a_1,a_2,\ldots,a_n\in\mathbb{Z}$, we define $(a_1,a_2,\ldots,a_n)$ to be the set of all integers of the form $a_1x_1+a_2x_2+\cdots+a_nx_n$ with $x_1,x_2,\ldots,x_n\in\mathbb{Z}$.
\end{definition}
\begin{remark}
Let $A = (a_1,a_2,\ldots,a_n)$. Notice that the sum and difference of two elements in $A$ are again in $A$. Also, if $a\in A$ and $r\in\mathbb{Z}$, then $ra\in A$, i.e., $A$ is an ideal in the ring $\mathbb{Z}$
\end{remark}
\begin{lemma}
If $a,b\in \mathbb{Z}$, then there is a $d\in \mathbb{Z}$ such that $(a, b)=(d)$
\end{lemma}
\begin{definition}
Let $a,b\in\mathbb{Z}$. An integer $d$ is called a greatest common divisor of $a$ and $b$ if $d$ is a divisor of both $a$ and $b$ and if every other common divisor of $a$ and $b$ divides $d$.
\end{definition}
\begin{remark}
The gcd of two numbers, if it exists, is determined up to sign.
\end{remark}
\begin{lemma}
Let $a,b\in\mathbb{Z}$. If $(a,b) = (d)$ then $d$ is a greatest common divisor of $a$ and $b$.
\end{lemma}
\begin{definition}
We say that two integers $a$ and $b$ are relatively prime if the only common divisors are $\pm1$, the units.
\end{definition}
It's fairly standard to use the notation $(a, b)$ for the greatest common divisor of $a$ and $b$. With this convention we can say that $a$ and $b$ are relatively prime if $(a,b)=1$.
\begin{proposition}
Suppose that $a\mid bc$ and that $(a,b)=1$. Then $a\mid c$.
\end{proposition}
\begin{corollary}
If $p$ is a prime and $p\mid bc$, then either $p\mid b$ or $p\mid c$.
\end{corollary}
\begin{corollary}
Suppose that $p$ is a prime and that $a,b\in\mathbb{Z}$. Then $\mbox{ord}_pab=\mbox{ord}_pa+\mbox{ord}_pb.$
\end{corollary}
\section{Class Notes 17-01-12}\label{PIDUFD}
\begin{definition}
A non-zero element in \R{} is called a unit if $\exists\, v\in \R$ such that $uv=1_\R$.
\end{definition}
\begin{definition}
Two element $a,b \in\R$ are said to be associative if $\exists\, a\in\R$ such that $a=bu$, denoted by $a\sim b$.
\end{definition}
\begin{definition}
A non-zero element $\pi$ in \R{} is said to be irreducible if $\pi$ is not a unit and if $a\mid \pi\Rightarrow$ $a$ is a unit or $a$ is associative of $\pi$.
\end{definition}
\begin{definition}
A non-zero element in $\R$ is said to be prime if $\pi$ is not a unit and $\pi\mid  ab\Rightarrow \pi\mid a$ or $\pi\mid b$, $\forall a,b\in\R.$
\end{definition}
\begin{proposition}
If $\pi$ is a prime, then $\pi$ is irreducible.
\end{proposition}
\begin{proof}
Let $\pi$ be a prime, suppose $a\mid \pi$, then $\pi=ab$ for some $b\in\R$. Thus $\pi\mid ab$ and by definition, $\pi\mid a$ or $\pi\mid b$.
\begin{itemize}
\item If $\pi\mid a$, then $a\sim \pi$.
\item If $\pi\mid b$, then $a\sim 1$.
\end{itemize}
\end{proof}
\begin{remark}
A irreducible is not necessary to be a prime.\\
Let $R = \Z[\sqrt{5}] = \{a+b\sqrt{-5}\ \mid \ a,b\in\Z\}\subset \C$. We have $$6=2\cdot3 = (1+\sqrt{-5})\cdot(1-\sqrt{-5}).$$We write $\pi = (1+\sqrt{-5})$ and claim that $2,3,\pi,\overline{\pi}$ are irreducibles but none of them are associative of each other.\\
We define the norm function $N:R\rightarrow \Z$, where $N(\al)=\al\overline{\al}$, i.e., if $\al = a+bi$, then $N(\al)=a^2+5b^2$. We notice that 
\begin{itemize}
\item If $\al>0$, then $N(\al)>0$.
\item $N(\al\be)=N(\al)N(\be)$.
\end{itemize}
\underline{Check: $2$ is irreducible:}\\
\underline{Find unit:} \\$N(uv) = N(1) = 1 = N(u)N(v) \Rightarrow N(u)=N(v) = 1$. But $a^2+5b^2 = 1\Rightarrow a=\pm1, b=0$.\\Suppose $2=\al\be$, then $4=N(2) = N(\al\be) = N(\al)N(\be)$.
\begin{enumerate}
\item If $N(\al) = 1, N(\be) = 4$\\
Then $\al$ is a unit $\Rightarrow$ $2$ is irreducible.
\item If $N(\al) = 2, N(\be) = 2$\\
Then $a^2+5b^2 = 2$ has no solution.
\end{enumerate}
\end{remark}
\begin{definition}
An UFD (Unique Factorization Domain) is an integral domain $R$ in which every non-zero element(up to unit) factors uniquely into a product of irreducibles.
\end{definition}
\begin{proposition}
Let $R$ be a domain in which factorization (of irreducibles) exists. Then \emph{$R$ is a UFD $\Leftrightarrow$ every irreducible in $R$ is prime.}
\end{proposition}
\begin{proof}\mbox{}\\
$``\Leftarrow":$ Let $a$ be an element of $R$ and $a\ne0$. If $a=\pi_1\pi_2\cdots\pi_n=\sigma_1\sigma_2\cdots\sigma_m$ are two factorizations. Since $\pi_1$ is prime, $\pi_1\mid \sigma_i$ for some $i$. By rearranging, we may assume $\pi_1\mid \sigma_1$, Thus $\pi_1\sim \sigma_1$. Repeating this process, we can conclude that the two factorizations are the same.
\notcomplete
\end{proof}
\begin{remark}
There are clearly rings such that no factorization exists. For example, consider the ring $\Z[2^{1/2},2^{1/4},2^{1/8},\ldots]\subset \R$. It's the smallest subring of $\R$ that contains $2^{1/2},2^{1/4},\ldots$.
\end{remark}
\begin{definition}
A ring $R$ is said to be noetherian if it satisfies any of the following equivalent conditions:
\begin{enumerate}
\item Any ascending chain of ideals in $R$ terminates.\\
Namely, $I_1\subset I_2\subset I_3\subset \cdots\Rightarrow I_n = I_{n+1}=\cdots$ for some $n$.
\item Any ideal $I$ in $R$ is finite generated.\\
Namely, $I=(a_1,\ldots,a_n)$ for some $n$.
\end{enumerate}
\end{definition}
\begin{proof}
\mbox{}\\\mbox{``1.\ $\Rightarrow$2.\ ": } Let $I$ be an ideal, if $I\ne0$, pick $a_1\in I, a_1\neq 0$, clearly $(a_1)\subset I$. 
If $(a_1)=I$, we are done, If not,
$\exists a_2\in I\backslash(a_1)\Rightarrow (a_1,a_2)\subset I $, this chain terminates.\\
\mbox{``1.\ $\Leftarrow$2.\ ": }Suppose $I_1\subset I_2\subset \ldots$ be an ascending ideal. Let $I = \cup I_n$, we claim that $I$ is an ideal.\\
Let $a,b\in I$, then there exists $n$ such that $a,b\in I_n$. Therefore $a+b\in I_n$, and $a+b\in I$. Let $a\in I$, then $a\in I_n$ for some $n$. Therefore $ra\in I_n\implies ra\in I$. Thus $I$ is an ideal. But $I=(a_1,\ldots,a_m)$, so there exists $n$, such that $a_1,\ldots,a_m\in I_n$. Thus $I=I_n$ and $I_n=I_{n+1}=\cdots$.
\end{proof}
\begin{exercise}
Suppose $R$ is a Noetherian domain, show $R$ admits factorizations.
\end{exercise}
\begin{proof}
If $b$ is not irreducible, then $b=ac$ or $(b)\subset(a)$\notcomplete
\end{proof}
\begin{definition}
A PID (Principle Ideal Domain) is a domain in which every ideal is generated by a single element.
\end{definition}
\begin{theorem}
Every PID is a UFD.
\end{theorem}
\begin{proof}
Let $R$ be a PID, then it's noetherian. So factorizations exist. So it suffices to show that every irreducible is a prime. Let $\pi$ be a irreducible in $R$. Suppose $\pi\mid ab$ and $a$ is not divided by $\pi$. We look at $I=(a,\pi)$, there exists $c\in R$, such that $I=(c)$. Thus we have $c\mid \pi, c\mid a$. So $c\sim 1$ or $c\sim \pi$. Since $c$ is not associative of $\pi$, $c$ is associative of $1$. But then $$1=ax+\pi y$$ for some $x,y\in R$. So $b=abx+\pi by$ or $\pi\mid b$.
\end{proof}
\section{Class Notes 17-01-17}
\begin{example}
\Z{} is a PID.
\end{example}
\begin{remark}
Any ideal $I\subset\Z$ is of the form of $n\Z$.
\end{remark}
\begin{proof}
$\forall I\subset \Z$, if $I=(0)$, we are done. If $I$ is not zero ideal, let $n$ be the smallest positive element in $I$. We claim: $I=n\Z$. Let $b\in I$, then $b=nq+r$, where $0\leq r<n$. But $r=b-nq\implies r\in I\implies r=0$. Therefore $b= nq$. 
\end{proof}
If $K$ is a field, let $R = k[x] = \mbox{polynomial in variable $x$ over the field $K$}$. What are the units in $R$? For arbitrary $f(x),g(x)\in K[x]$, if $f(x)g(x)=1$, we claim that $f(x), g(x)$ must be constant polynomial. For if we write $f(x)=a_nx^n+a_{n-1}x^{n-1}+\cdots$, $g(x)=b_mx^m+b_{m-1}x^{m-1}+\cdots$. Then $f(x)g(x) = a_nb_m x^{m+n}+\cdots$. Since $a_n\ne0,b_m\ne0$ and \emph{$K$ is an integral domain}, we have $a_nb_m\ne0$. Therefore $$\deg{f(x)g(x)} = \deg{f(x)}+\deg{g(x)}.$$ We then apply this conclusion to $f(x)g(x) = 1$ and get $\deg{f(x)}\deg{g(x)} = \deg{1} = 0$, thus $f(x), g(x)$ must be constant.
\begin{remark}
Whether a polynomial is irreducible depends on the field. For example, if $x^2+1\in \R[x]$, then it's irreducible (why?). But if $x^2+1\in \C[x]$, then it's reducible (why?).
\end{remark}
\underline{Division Algorithm:} Let $f(x), g(x)\in K[x], g(x)\ne 0$, then there exists $q(x),r(x)\in K[x]$, such that $$f(x)=g(x)q(x)+r(x),$$where $r(x)=0$ or $0\leq\deg{r(x)}<\deg{g(x)}$. \\
Using this fact, we have the following theorem.
\begin{theorem}\label{kxpid}
$K[x]$ is a PID.
\end{theorem}
\begin{proof}
For all ideal $I\in K[x]$, if $I=(0)$, we are done. If $I\ne(0)$, let $g(x)\in I$ be the polynomial of least degree, let $f(x)\in I$, then $$f(x)=g(x)q(x)+r$$with $r=0$ or $0\leq\deg{r(x)}<\deg{g(x)}$ by division algorithm. But then $r(x)=0$, for otherwise $r(x)$ will be a polynomial whose degree is less than $g(x)$. Therefore $f(x)=g(x)q(x)$, $f(x)\in (g(x))$.
\end{proof}
\begin{definition}
A domain $R$ is said to be an Euclidean domain if there exists a function $\lambda: \R\setminus\{0\}\rightarrow\Z^{\ge0}$, such that given $a,b\in R, b\ne0$, there exist $q,r\in R$ such that $a=qb+r$ and either $r=0$ or $0\le\lambda(r)<\lambda(b)$.
\end{definition}
\begin{example}
$R=\Z[i]$ is an Euclidean domain.
\end{example}
\begin{proof}
Let $N(\al)=\al\overline{\al} = a^2+b^2$ (if $\al = a+bi$). Let $\al,\be\in R$, $\be\ne0$, we have
$$\frac{\al}{\be} = \frac{a+bi}{c+di} = \frac{ac+bd}{c^2+d^2}+\frac{bc-ad}{c^2+d^2}i = r+si. (r,s\in\Q)$$Let $m+ni\in\Z[i]$ be the closest element to $r+si$. We denote $r'=r-m,s'=s-n$, then $\frac{\al}{\be} = r+si = m+ni+r'+s'i$, or 
$$\al = \be(m+ni)+\be(r'+s'i),$$where $(m+ni)\in\Z[i]$ and $\be(r'+s'i)\in\Z[i]$, we remain to show that $N(\be(r'+s'i))<N(\be)$. This is the case because 
\begin{align*}
N(\be(r'+s'i))&= N(\be)N(r'+s'i)\\&\leq N(\be)(\frac{1}{4}+\frac{1}{4})\\&<N(\be)
\end{align*} We are done.
\end{proof}
The Natural question is what are the units in $\Z[i]$? Does a prime in \Z{} still a prime in $Z[i]$? To answer the first question, we assume $u$ is a unit in $\Z[i]$. Then by definition there exists some $v$ such that $uv= 1$. But then $1=N(1) = N(uv) = N(u)N(v)\implies N(u) = 1$. Thus the only possible values of $u$ is $\pm1,\pm i$. We also check they are actually units. Now, to answer the second question, we try some small cases. We look at $5,7,11$ and $13$.
\begin{example}
If $5=ab$, $a,b\in \Z[i]$, then $25=N(5) = N(ab) = N(a)N(b)\implies N(a) = 5$. So $a$ can only be $\pm1\pm2i$ or $\pm2\pm i$. We try by hand and find $5= (2+i)(2-i)$ is a factorization, so $5$ is not a prime. 
\end{example} 
\begin{example}
If $7=ab$, $a,b\in \Z[i]$, then $49=N(5) = N(ab) = N(a)N(b)\implies N(a) = 7$. We try by hand and find no factorization, so $7$ is a prime.
\end{example}
Use the same method, we find $5,13$ are not prime while $7,11$ are prime.
\begin{remark}\mbox{}Obervation:
\begin{enumerate}
	\item If $p\con{1}{4}$, then $p = \pi\overline{\pi}$, where $\pi$ is a irreducible.
	\item If $p\con{3}{4}$, then $p$ remains prime.
	\item If $p=2$, $2=(1+i)(1-i) = (-i)(1+i)^2$ (ramification).
\end{enumerate}
\end{remark}
\begin{remark}
Let $R = \Z[\omega]$, where $\omega$ is a primitive cube root of $1$, then $R$ is a Euclidean domain.
\end{remark}
\section{Unique Factorization in $k[x]$}
In this section we consier the ring $k[x]$ of polynomials with coefficients in a field $k$. If $f,g\in k[x]$, we say that $f$ divides $g$ if there is an $h\in k[x]$ such that $g= fh$.\\
If $\deg{f}$ denotes the degree of $f$, we have $\deg{fg}= \deg{f}+\deg{g}$ (why? Because a field $k$ is necessarily an integral domain). nonzeros constants are the units of $k[x]$. A nonconstant polynomial $p$ is said to be irreducible if $q\mid p\implies$ $q$ is either a constant or a constant times $p$.
\begin{lemma}\label{l1}
Every nonconstant polynomial is the product of irreducible polynomials.
\end{lemma}
\begin{proof}
Simply by induction.
\end{proof}
\begin{definition}
A polynomial $f$ is called monic if its leading coefficient is $1$.
\end{definition}
\begin{definition}
Let $p$ be a monic irreducibe polynomial. We define $\ord{p}{f}$ to be the integer $a$ defined by the property that $p^a\mid f$ but that $p^{a+1}\nmid f$.
\end{definition}
\begin{remark}
$\ord{p}{f} = 0$ iff $p\nmid f$.
\end{remark}
\begin{theorem}
Let $f\in k[x]$. Then we can write$$f= c\prod_pp^{a(p)},$$ where the product is over all monic irreducible polynomials and $c$ is a constant. The constant $c$ and the exponents $a(p)$ are uniquely determined by $f$; in fact, $a(p)= \ord{p}{f}$.
\end{theorem}
The existence of such a product follows immediately from Lemma \ref{l1}. The uniqueness part is more difficult and will be postponed.
\begin{lemma}
Let $f,g\in k[x]$. If $g\ne 0$, there exist polynomials $h,r\in k[x]$ such that $f=hg+r$, where either $r=0$ or $r\ne0$ and $\deg{r}\le \deg{g}$.
\end{lemma}
\begin{proof}
If $g\mid f$, we are done. If $g\nmid f$, let $r= f-hg$ be the polynomial of least degree among all polynomials of the form $f-lg$ with $l\in k[x]$. We claim that $\deg{r}<\deg{g}$. If not, let the leading term of $r$ be $ax^d$ and that $g$ be $bx^m$. Then $r-\frac{a}{b}x^{d-m}g(x)= f-(h+\frac{a}{b}x^{d-m})g$ has smaller degree than $r$ and is of the given form. This is a contradiction.
\end{proof}
\begin{lemma}\label{gcdfgd}
Given $f,g\in k[x]$ there is a $d\in k[x]$ such that $(f,g)=(d)$.
\end{lemma}
\begin{proof}
See Theorem \ref{kxpid}.
\end{proof}
\begin{definition}
Let $f,g\in k[x]$. Then $d\in k[x]$ is said to be a greatest common divisor of $f$ and $g$ if $d$ divides $f$ and $g$ and every common divisor of $f$ and $g$ divides $d$.
\end{definition}
\begin{remark}
Notice that the greatest common divisor of two polynomials is determined up to multiplication by a constant. If we require it to be monic, it is uniquely determined and we may speak of the greatest common divisor.
\end{remark}
\begin{lemma}
Let $f,g\in k[x]$ By lemma \ref{gcdfgd} there is a $d\in k[x]$ such that $(f,g)=(d)$. $d$ is the greatest common divisor of $f$ and $g$.
\end{lemma}
\begin{proof}
Since $f\in (d)$ and $g\in(d)$ we have $d\mid f$ and $d\mid g$. Suppose that $h\mid f$ and that $h\mid g$. Then $h$ divides every elements in $(f,g)=(d)$. In particular $h\mid d$, we are done.
\end{proof}
\begin{definition}
Two polynomial $f$ and $g$ are said to be relatively prime if the only common divisor of $f$ and $g$ are constants. In other words, $(f,g) = (1)$.
\end{definition}
\begin{proposition}
If $f$ and $g$ are relatively prime and $f\mid gh$, then $f\mid h$.
\end{proposition}
\begin{corollary}
If $p$ is an irreducible polynomial and $p\mid fg$, then $p\mid g$ or $p\mid g$. 
\end{corollary}
\begin{corollary}
If $p$ is a monic irreducible polynomial and $f,g\in k[x]$, we have $$\ord{p}{fg}=\ord{p}{f}+\ord{p}{g}.$$
\end{corollary}
Using these tools, we can prove the uniqueness of factorizaion.
\section{Unique Factorizaion in a Principal Ideal Domain}
For this section, we mostly refer to Section \ref{PIDUFD} and supply some details. 
\chapter{Congruence}
\section{Class Notes 17-01-19}
\begin{definition}
We write $a\con{b}{p}$, if $p\mid (a-b)$.
\end{definition}
\begin{remark}
To solve $ax\con{b}{m}$ in \Z{} is the same to solve $[a]x=[b]$ in $\Z/m\Z$.
\end{remark}
We now try to solve the equation $a\con{b}{m}$.
\begin{proposition}
A necessary and sufficient condition for this equation to have solutions is $d\mid b$, where $d=(a,m)$ is the gcd of $a$ and $m$.
\end{proposition}
\mbox{}\\\underline{Think About: }
$ax\con{1}{m}$ has solutions is equivalent to $(a,m) = 1$.
\begin{proof}\mbox{}\\
$``\Rightarrow'':$ If we have some solution $x_0$ such that $ax_0\con{1}{m}$. Then $ax_0=1+mt$ so that $(a,m)= 1$.\\
$``\Leftarrow'': $ If $(a,m) = 1$, then there exists $x_0, t$ such that $1=ax_0-mt$, so $ax_0\con{1}{m}$.
\end{proof}
\begin{remark}
In $\Z/m\Z$, $[a]x\equiv[1]$ implies that $[a]$ is a unit. 
\end{remark}
\begin{definition}
$\phi(m) =$ \# of units in $\Z/m\Z$. 
\end{definition}
We give a few example:\quad
\begin{tabular}{|c|c|c|c|c|c|}\hline
$m$ & $1$ & $2$ & $3$ & $4$ & $5$\\\hline
$\phi(m)$ & $1$ & $1$ & $2$ & $2$ & $4$\\\hline
\end{tabular}\\[6pt]
Now we give the formal proof of our proposition.
\begin{proof}
Suppose $x_0$ is a solution, then there exist $t$ such that $$ax_0=b+mt,$$ Since $(a,m)\mid a$, $(a,m)\mid m$, we have $(a,m)\mid b$. Conversely, suppose $(a,m)\mid b$, we may write $b$ as $b= (a,m)b'$. Similarly, $a=(a,m)a'$ and $m=(a,m)m'$ with $(a',m')=1$. Denote $d:=(a,m)$, then $da'x\con{db'}{dm'}$, $a'x\con{b'}{m'}$. Since $(a',m')=1$, $a'x\con{b'}{m'}$ has solutions.
\end{proof}
\begin{remark}
According to the proof, we will have $d=(a,m)$ solutions.
\end{remark}
Now we want to introduce \emph{Chinese Remainder Theorem in $\Z$}. We want to solve a system of congruence equations. Namely, we are looking at the system
\begin{align*}
x&\con{a_1}{m_1}\\
x&\con{a_2}{m_2}\\
&\vdots\\
x&\con{a_n}{m_n}
\end{align*}
where $m_i$ are pairwise coprime.
\begin{theorem}[Chinese Remainder Theorem]\mbox{}\\
The system always admits solutions.
\end{theorem}
 We notice that if $x_0$ is a solution to the system, so does $x=km_1m_2\cdots m_n+x_0$, $k\in\Z$. So the system will have infinitely many solutions. The sketch of the proof is as followed.\\ Suppose we can solve the system
\begin{align*}
x_i&\con{1}{m_i}\\
x_i&\con{0}{m_j}\quad\forall j\ne i
\end{align*}
then $x=a_1x_1+a_2x_2+\cdots+a_nx_n$ is a solution for the original system. But why does the system even have a solution?\\
Consider the following system as an example,
\begin{align*}
x&\con{1}{m_1}\\
x&\con{0}{m_2}\\
&\vdots\\
x&\con{0}{m_n}
\end{align*}
We know that since $m_i$ are coprime, $(m_1,m_2m_3\cdots m_n) = 1$.
\begin{align*}
\Rightarrow &\ \exists\;c,d_1,\,\mbox{s.t.}\  cm_1+d_1m_2m_3\cdots m_n = 1\\
\Rightarrow &\ x=d_1m_2m_3\cdots m_n\ \mbox{is a solution}
\end{align*}
\begin{remark}
If there are two solutions for the system, say $x$ and $y$, then $$x-y\con{0}{m_1m_2\cdots m_n}\implies x\con{y}{m_1m_2\cdots m_n}.$$ Namely, the solution is unique up to a multiple of $m_1m_2\cdots m_n$. 
\end{remark}
In order to generalize CRT, we need some background.\\
Suppose $R, S$ are two rings, then $R\times S := \{(r,s), r\in R, s\in S\}$. We also define sum and product on $R\times S$, namely,
\begin{align*}
(a,b)+(c,d) &=(a+c, b+d),\\
(a, b)\cdot (c,d) &= (ac, bd).
\end{align*}
We can check that $R\times S$ is actually a ring. The projection maps are ring homomorphisms, i.e., there exist projection maps $E_S, E_R$, \begin{align*}
E_S: R\times S&\rightarrow S\\
E_R: R\times S&\rightarrow R
\end{align*} But there doesn't exist any homomorphism from $S$ or $R$ to $R\times S$.\\
We know that for a ring homomorphism $\phi: R\rightarrow S$, $\ker{\phi} = \{x\in R, \phi(x) = 0\}$ is an ideal. For ring homomorphism $\Z\rightarrow \Z/m\Z$, it's kernal is exactly the ideal $m\Z$. So in fact, what CRT in \Z{} says is that the ring homomorphism $$f:\Z\rightarrow \Z/m_1\Z\times\Z/m_2\Z\times\cdots\times\Z/m_n\Z,$$ or $$a\mapsto ([a]_{m_1},\ldots,[a]_{m_n})$$is surjective.
\begin{diagram}
R & \rTo^\phi & S\\
\dTo_f & \ruDashto_{\tilde{\phi}}\\
R/I
\end{diagram}
For a ring homomorphism $\phi: R\rightarrow S$, $I = \ker\phi$,
\begin{itemize}
\item $\phi$ is injective if and only if $\ker\phi = \{0\}$.
\item There exists a unique ring homomorphism $\tilde{\phi}: R/I\rightarrow S$, or $\tilde{\phi}: [a]\mapsto \phi(a)$ such that the diagram commutes. $\tilde{\phi}$ is also well defined, for if $[a]= [b]$, then we have \begin{align*}
[a]=[b] &\Rightarrow (a-b)\in I\\
		&\Rightarrow \phi(a-b) = 0\\
		&\Rightarrow \phi(a) = \phi(b).
\end{align*} 
\end{itemize}
Now, let $R = \Z$, $S = \Z/m_1\Z\times \Z/m_2\Z\times \cdots\times \Z/m_n\Z$. Let $m = m_1m_2\cdots m_n$, then $\ker\phi = \Z/m\Z$, we have the following diagram.
\begin{diagram}
\Z & \rTo^\phi & \Z/m_1\Z\times \Z/m_2\Z\times \cdots\times \Z/m_n\Z\\
\dTo_f & \ruTo_{\tilde{\phi}}\\
\Z/m\Z
\end{diagram}
Notice that $\tilde{\phi}$ is an isomorphism.\\
We have the natural question that what are the units in $R$ and $S$? Let $U(R)$ denote the set of units of the ring $R$, then $U(R\times S) = U(R)\times U(S)$.
We thus have a branch of corollaries.
\begin{corollary}
$U(\Z/m\Z) \cong U(\Z/m_1\Z)\times \cdots\times U(\Z/m_n\Z)$.
\end{corollary}
\begin{corollary}
$\phi(m) = \phi(m_1)\phi(m_2)\cdots\phi(m_n)$.
\end{corollary}
\begin{corollary}
If $m = p_1^{\gamma_1}p_2^{\gamma_2}\cdots p_s^{\gamma_s}$, then 
\begin{align*}
\phi(m) &= \phi(p_1^{\gamma_1}p_2^{\gamma_2}\cdots p_s^{\gamma_s})\\
&=\phi(p_1^{\gamma_1})\phi(p_2^{\gamma_2})\cdots\phi(p_s^{\gamma_s}), 
\end{align*}
with $\phi(p_i^{\gamma_i}) = p_i^{\gamma_i} -p_i^{\gamma_i-1}$.
\end{corollary}
\begin{corollary}
$$\sum_{d\mid n}\phi(d) = n$$
\end{corollary}
The proof is simply use the fact that the statement is true for primes, and every element of \Z can be factorized as a product of primes.
\begin{proof}
We claim that if the statement is true for $m,n\,((m,n) = 1)$, then it's true for $mn$.
\begin{align*}
\sum_{d\mid mn} \phi(d) &=\sum_{d_1\mid m,d_2\mid n} \phi(d_1d_2)\\
&= \sum_{d_1\mid m}\sum_{d_2\mid n} \phi(d_1)\phi(d_2)\\&=(\sum_{d_1\mid m}\phi(d_1))(\sum_{d_2\mid n}\phi(d_2))\\
&= m\cdot n.
\end{align*}
\end{proof}
\section{Class Notes 17-01-24}
Suppose $I,J\subset R$ are two ideals, how to make new ideals with $I,J$? Evidently, $I\cap J$ and $I+J$ are ideals. Also,
$$I\cdot J:=\{\sum a_i b_i, a_i\in I, b_i\in J\}\subset I\cap J$$ is an ideal. 
\begin{example}
Let $I=m\Z$,$J=n\Z$. then we have \quad\begin{tabular}{|c|c|c|}\hline
$I+J$ & $I\cap J$ & $I\cdot J$\\\hline
$((m,n))$ & $([m,n])$ & $mn\Z$\\\hline
\end{tabular}.
\end{example}
\begin{definition}
We say two ideals $I,J$ are coprime if $I+J=(1)$.
\end{definition}
\begin{remark}
If $I, J$ are coprime, then $I\cap J = I\cdot J$.
\end{remark}
\begin{proof}
For some $x\in I\cap J$, since $I,J$ are coprime, there exists some $a\in I, b\in J$ such that $a+b = 1$. But then $a\cdot x + x\cdot b = x\in I\cdot J$. So $I\cap J \subset I\cdot J$. The other direction is obvious.
\end{proof}
\begin{theorem}[Generalized Chinese Remainder Theorem]
Let $I_1, I_2, \ldots, I_n$ be pairwise coprime ideals in $R$, then the map
$$\phi: R\rightarrow R/I_1\times \ldots \times R/I_n$$
\begin{itemize}
\item[1)] is surjective
\item[2)] has $\ker\phi = I_1I_2\ldots I_n = I_1\cap I_2\cap \ldots \cap I_n$
\end{itemize}
\end{theorem}
\begin{lemma}
We first look at $n=2$ case. If $I,J$ are coprime ideals in $R$, then the map
$$\phi: R\rightarrow R/I \times R/J$$
\begin{itemize}
\item[1)] is surjective.
\item[2)] has $\ker\phi = I\cap J = IJ$.
\end{itemize}
\end{lemma}
\begin{proof}
It's enough to solve the system of congrence
\begin{align*}
x&\con{1}{I}\\
x&\con{0}{J}
\end{align*} and \begin{align*}
y&\con{0}{I}\\
y&\con{1}{J}
\end{align*}
Since $I,J$ are coprime, there exists $c\in I, d\in J$ such that $c+d = 1$. $c, d$ is the solution to our two systems.
\end{proof}
\begin{lemma}
$I_1$ is coprime to $I_2I_3\cdots I_n$.
\end{lemma}
\begin{proof}
There exist \begin{align*}
a_2+b_2 &= 1\\
a_3+b_3 &= 1\\
\ldots\\
a_n+b_n &=1,
\end{align*}
$a_i\in I_1$, $b_j\in I_j$.\\
Then \begin{align*}
b_2b_3\ldots b_n &= (1-a_2)\ldots (1-a_n)\\
&= 1 + a, 
\end{align*}
where $a\in I_1$.\\
By $n=2$ case 
\begin{diagram}
R & \rTo & R/I_1\times R/(I_2\times I_3\times\cdots\times I_n)\\
\dTo & \ruTo\\
R/I_1\times R/I_2\times \cdots\times R/I_n\\
\end{diagram}
\end{proof}
Let us denote $U(R)$ by $R^{\times}$. Note that $\phi(n)=\abs{(\Z/n\Z)^{\times}}$. We now want to look at the structure of $(\Z/n\Z)^{\times}$. We first develop some background in abstract algebra.
\begin{theorem}[Lagrange Theorem]
	Let $G$ be a finite group, $H\subset G$ is a subgroup, then the order of $H$ divides the order of $G$, i.e.,$$|H|\mid|G|$$
\end{theorem}
\begin{proof}
	Take two cosets in $H$, $Ha$ and $Hb$. They are equal or disjoint. So $$|G| = |H|\cdot\hbox{\# of cosets}$$
\end{proof}
\begin{definition}
	If $a\in G$, then $o(a) = \hbox{smallest positive integer}$ $d$ such that $$a^d = 1$$is called the order of the element $a$.
\end{definition}
\begin{corollary}
	$\forall a\in G$, we have $o(a)\mid |G|$.
\end{corollary}
\begin{proof}
	$\langle a\rangle:=\{1,a,\ldots, a^{d-1}\}$ is the subgroup generated by $a$. Then $\langle a\rangle\subset G\Rightarrow d\mid |G|$.
\end{proof}
\begin{corollary}
	$a^{|G|} = 1$.
\end{corollary}
\begin{corollary}
	If $n\ge1$, $(a,n) = 1$, then $a^{\phi(n)}\con{1}{n}$.
\end{corollary}
\begin{proof}
	$(a,n) = 1 \Rightarrow a\rightarrow[a]$ is a unit in $\Z/n\Z$, i.e., $[a]\in(\Z/n\Z)^{\times},|(\Z/n\Z)^{\times}| = \phi(n).$
	$\Rightarrow [a]^{\phi(n)}=1$ in $(\Z/n\Z)^{\times}$,
	i.e., $a^{\phi(n)}\con{1}{n}$.
\end{proof}
\begin{exercise}
	Find the last $3$ digits of $3^{1203}$.
\end{exercise}
\begin{proof}
	$\phi(1000) = \phi(2^3 5^3) = (8-4)(125-25) = 400$. So $3^{400}\con{1}{1000}$. The last three digits are then $027$.
\end{proof}
We now look at the structure of $(\Z/p\Z)^{\times}$, where $p$ is a prime.
\begin{theorem}
	$(\Z/p\Z)^{\times}$ is cyclic.
\end{theorem}
We do some checking, let $p=5,7,11,13$. For $p=11$, we find that $2,3,7,9$ are $\Z/11\Z$'s generator.
\begin{lemma}
	Let $a\in G$ be an element of order $d$, then the order of $a^m$ is $\frac{d}{(d,m)}$.
\end{lemma}
\begin{proof}
	Let $(d,m) = b$, we then have $d = bd', m = bm'$, where $(d',m') = 1$. We claim that $o(a^m) = d'$. For $(a^m)^{d'} \cong a^{bm'd'}\cong a^{dm'}\cong (a^d)^{m'}\cong 1$. Suppose $(a^m)^{l} = 1\Rightarrow a^{ml} = 1\Rightarrow d\mid ml \Rightarrow bd'\mid bm'l \Rightarrow d'\mid m' l\Rightarrow d'\mid l$.
\end{proof}
\begin{corollary}
	If $G$ is cyclic of order $d$, then the number of generators of $G$ is $\phi(d)$.
\end{corollary}
\section{Class Notes 17-01-26}
\begin{theorem}
	$(\Z/p\Z)$ is a field.
\end{theorem}
\begin{proof}
	If $[a]\ne0 \Rightarrow (p,a) = 1 \Rightarrow \exists x,y\ s.t.\ px+ay = 1\Rightarrow [a][y] = [1]$.
\end{proof}
\begin{theorem}
	Let $K$ be a field, let $G$ be a finite subgroup of $K$, then $G$ is cyclic.
\end{theorem}
\begin{lemma}
	Let $f(x)\in K[x]$ be any non-zero polynomial. Then the number of roots of $f$ in $K$ is elss or equal to $\deg{f}$
\end{lemma}
\begin{proof}
If $f(x)$ has no root, we are done. If $f(x)$ has some roots, say $\al$ is a root, then $$f(x) = (x-\al)g(x)+ r(x),\quad r(x) = 00$$So $f(x) = (x-\al)g(x)$. By induction the lemma holds.
\end{proof}
We can then prove the theorem.
\begin{proof}
	Let $K$ be a field. Let $G\subset K^{\times}$ be a finite subgroup of order $n$. $G\subset \{\hbox{roots of} x^n-1\}\Rightarrow G = \{\hbox{roots of} x^n -1\}$. Any element in $G$ has order dividing by $n$ for every divisor $d$ of $n$. Let $\Sigma_d = \{a\in G, o(a) = d\}$, then $$G=\sqcup_{d\mid n} \Sigma_d, \quad n=|G| = \sum_{d\mid n}|\Sigma_d|.$$
	We claim: $|\Sigma_d| = 0$ or $\phi(d)$.\\
	If $\Sigma_d = \emptyset\Rightarrow |\Sigma_d| = 0$. Suppose $\Sigma_d\ne \emptyset\Rightarrow \exists a\in G, \hbox{s.t.} o(a) = d$. Let $H =\langle a\rangle = \{1,a,\ldots, a^{d-1}\}\subset G.$
	i.e.,\begin{align*}\Sigma_d &=\hbox{set of elements with order } d\\
	&= \hbox{all elements of } H
	\end{align*}.
	$\Rightarrow |\Sigma_d| = \phi(d)$. Then $$n=\sum_{d\mid n}|\Sigma_d|\le \sum_{d\mid n}\phi(d) = n$$
	$\Rightarrow |\Sigma_d| = \phi(d), \forall d\mid n$. In particular $|\Sigma_n| = \phi(n)\Rightarrow G$ is cyclic.
\end{proof}
We then want to discuss the structure of $(\Z/p^{\gamma}\Z)^{\times}$
\begin{theorem}
	If $p$ is an odd prime, then $(\Z/p^{\gamma}\Z)^{\times}$ is cyclic.
\end{theorem}
\begin{proof}
	Since $\Z/p^{\gamma}\Z\rightarrow\Z/p\Z$ is surjective, $(\Z/p^{\gamma}\Z)^{\times}\rightarrow(\Z/p\Z)^{\times}$ is surjective. Let us denote $G:=(\Z/p^{\gamma}\Z)^{\times}$, $H:=(\Z/p\Z)^{\times}$, and let $K$ be the kernal of $G\rightarrow H$, i.e.,
	 $$K=\{[x]\in G, x\con{1}{p}\}.$$
	Note we have $|G|=p^{\gamma -1 }(p-1), |H| = p-1$. So we have $|K| = \dfrac{|G|}{|H|} = p^{\gamma -1}$. We will show $K$ is cyclic by explicitly constructing a system. We consider the cyclic group generated by $1+ap$, where $a\con{0}{p}$. We know that 
	$$(1+ap)^{p^{\gamma-1}}\con{1}{p^{\gamma}},$$want however 
	$$(1+ap)^{p^{\gamma-2}}\ncon{1}{p^{\gamma}}.$$
	\begin{lemma}\label{yi}
		Let $p$ be any prime, $a,b\in\Z,\gamma\ge1$. If $a\con{b}{p^{\gamma}}$, then $a^p\con{b^p}{p^{\gamma+1}}$.
	\end{lemma}
	\begin{proof}
		First notice that for $1\le i \le p-1$, $\comb{p}{i}$ is divided by $p$, then
		\begin{align*}
			a=b+p^\gamma t\ &\Rightarrow\ a^p=(b+p^\gamma t)^p\\
			&\Rightarrow\ a^p=b^p+\sum\limits_{i=1}^{p-1}\comb{p}{i}b^i (p^\gamma t)^{p-1} + (p^{\gamma}t)^{p}.\\
			&\Rightarrow\ a^p\con{b^p}{p^{\gamma+1}}
		\end{align*}
	\end{proof}
	We then prove the following lemma,
	\begin{lemma}
		$(1+ap)^{p^{\gamma-2}}\con{1+ap^{\gamma-1}}{p^{\gamma}}$
	\end{lemma}
	\begin{proof}
		We induction on $\gamma$.\\When $\gamma=1$, the statement is trivially true. Assume the statement is true for $\gamma$, check for $\gamma+1$.\\
		We know $$(1+ap)^{p^{\gamma-2}}\con{1+ap^{\gamma -1}}{p^\gamma},$$and we want to show $$(1+ap)^{p^{\gamma-1}}\con{1+ap^{\gamma}}{p^{\gamma+1}}$$By lemma \ref{yi}, \begin{align*}(1+ap)^{p^{\gamma-1}}&\con{(1+ap^{\gamma-1})^p}{p^{\gamma+1}}\\
		&= 1+ p\cdot ap^{\gamma-1}+\sum\limits_{i=2}^{p-1}\comb{p}{i}(ap^{\gamma-1})^i+a^pp^{p(\gamma-1)}\\
		&\con{1+ap^{\gamma}}{p^{\gamma+1}}
		\end{align*}
		So the statement holds for $\gamma+1$.
	\end{proof}
\end{proof}
\section{Class Notes 17-01-31}
Last class we have prove that if $n=p$ os a prime, then $(\Z/p\Z)^{\times}$ is cyclic, and if $n$ is odd, $n=p^r$, $(\Z/p\Z)^{\times}$ is cyclic.\notcomplete
\mbox{}\\
Let $p$ is an odd prime, $(a,p) = 1$, is $a$ a square modulo $p$? We try $a=-1$ for $p=5,13,\ldots$. We have the following proposition.
\begin{proposition}
	$-1$ is a square modulo $p$ $\Longleftrightarrow$ $p\con{1}{4}$.
\end{proposition}
\begin{definition}
	We introduce the legendre symbol 
	\begin{equation*}
	\leg{a}{p}=\begin{cases}
	1 & \text{if $a$ is a square modulo $p$}\\
	-1 & \text{otherwise}
	\end{cases}
	\end{equation*}
\end{definition}
We have the following proposition.
\begin{proposition}\mbox{}\label{poss}
	\begin{enumerate}
		\item
		$\leg{a}{p} = \leg{b}{p}$ if $a\con{b}{p}$
		\item
		$\leg{a}{p}\con{a^{\frac{p-1}{2}}}{p}$
		\item\label{pos-3}
		$\leg{ab}{p} = \leg{a}{p}\leg{b}{p}$
	\end{enumerate}
\end{proposition}\mbox{}\\
The proof of Proposition \ref{poss}.\ref{pos-3} is as followed.
\begin{proof}
Let $g$ be a generator, then  $\langle\,g\,\rangle = \{1,g,g^2,\ldots,g^{p-1} \}$, $1,g^2,g^4,\ldots$
\end{proof}
\end{document}
































